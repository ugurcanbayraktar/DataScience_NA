% Options for packages loaded elsewhere
\PassOptionsToPackage{unicode}{hyperref}
\PassOptionsToPackage{hyphens}{url}
%
\documentclass[
]{article}
\title{Statistics of earthquake hazards in Turkey and comparison with
the world}
\author{Team NA}
\date{1/30/2022}

\usepackage{amsmath,amssymb}
\usepackage{lmodern}
\usepackage{iftex}
\ifPDFTeX
  \usepackage[T1]{fontenc}
  \usepackage[utf8]{inputenc}
  \usepackage{textcomp} % provide euro and other symbols
\else % if luatex or xetex
  \usepackage{unicode-math}
  \defaultfontfeatures{Scale=MatchLowercase}
  \defaultfontfeatures[\rmfamily]{Ligatures=TeX,Scale=1}
\fi
% Use upquote if available, for straight quotes in verbatim environments
\IfFileExists{upquote.sty}{\usepackage{upquote}}{}
\IfFileExists{microtype.sty}{% use microtype if available
  \usepackage[]{microtype}
  \UseMicrotypeSet[protrusion]{basicmath} % disable protrusion for tt fonts
}{}
\makeatletter
\@ifundefined{KOMAClassName}{% if non-KOMA class
  \IfFileExists{parskip.sty}{%
    \usepackage{parskip}
  }{% else
    \setlength{\parindent}{0pt}
    \setlength{\parskip}{6pt plus 2pt minus 1pt}}
}{% if KOMA class
  \KOMAoptions{parskip=half}}
\makeatother
\usepackage{xcolor}
\IfFileExists{xurl.sty}{\usepackage{xurl}}{} % add URL line breaks if available
\IfFileExists{bookmark.sty}{\usepackage{bookmark}}{\usepackage{hyperref}}
\hypersetup{
  pdftitle={Statistics of earthquake hazards in Turkey and comparison with the world},
  pdfauthor={Team NA},
  hidelinks,
  pdfcreator={LaTeX via pandoc}}
\urlstyle{same} % disable monospaced font for URLs
\usepackage[margin=1in]{geometry}
\usepackage{color}
\usepackage{fancyvrb}
\newcommand{\VerbBar}{|}
\newcommand{\VERB}{\Verb[commandchars=\\\{\}]}
\DefineVerbatimEnvironment{Highlighting}{Verbatim}{commandchars=\\\{\}}
% Add ',fontsize=\small' for more characters per line
\usepackage{framed}
\definecolor{shadecolor}{RGB}{248,248,248}
\newenvironment{Shaded}{\begin{snugshade}}{\end{snugshade}}
\newcommand{\AlertTok}[1]{\textcolor[rgb]{0.94,0.16,0.16}{#1}}
\newcommand{\AnnotationTok}[1]{\textcolor[rgb]{0.56,0.35,0.01}{\textbf{\textit{#1}}}}
\newcommand{\AttributeTok}[1]{\textcolor[rgb]{0.77,0.63,0.00}{#1}}
\newcommand{\BaseNTok}[1]{\textcolor[rgb]{0.00,0.00,0.81}{#1}}
\newcommand{\BuiltInTok}[1]{#1}
\newcommand{\CharTok}[1]{\textcolor[rgb]{0.31,0.60,0.02}{#1}}
\newcommand{\CommentTok}[1]{\textcolor[rgb]{0.56,0.35,0.01}{\textit{#1}}}
\newcommand{\CommentVarTok}[1]{\textcolor[rgb]{0.56,0.35,0.01}{\textbf{\textit{#1}}}}
\newcommand{\ConstantTok}[1]{\textcolor[rgb]{0.00,0.00,0.00}{#1}}
\newcommand{\ControlFlowTok}[1]{\textcolor[rgb]{0.13,0.29,0.53}{\textbf{#1}}}
\newcommand{\DataTypeTok}[1]{\textcolor[rgb]{0.13,0.29,0.53}{#1}}
\newcommand{\DecValTok}[1]{\textcolor[rgb]{0.00,0.00,0.81}{#1}}
\newcommand{\DocumentationTok}[1]{\textcolor[rgb]{0.56,0.35,0.01}{\textbf{\textit{#1}}}}
\newcommand{\ErrorTok}[1]{\textcolor[rgb]{0.64,0.00,0.00}{\textbf{#1}}}
\newcommand{\ExtensionTok}[1]{#1}
\newcommand{\FloatTok}[1]{\textcolor[rgb]{0.00,0.00,0.81}{#1}}
\newcommand{\FunctionTok}[1]{\textcolor[rgb]{0.00,0.00,0.00}{#1}}
\newcommand{\ImportTok}[1]{#1}
\newcommand{\InformationTok}[1]{\textcolor[rgb]{0.56,0.35,0.01}{\textbf{\textit{#1}}}}
\newcommand{\KeywordTok}[1]{\textcolor[rgb]{0.13,0.29,0.53}{\textbf{#1}}}
\newcommand{\NormalTok}[1]{#1}
\newcommand{\OperatorTok}[1]{\textcolor[rgb]{0.81,0.36,0.00}{\textbf{#1}}}
\newcommand{\OtherTok}[1]{\textcolor[rgb]{0.56,0.35,0.01}{#1}}
\newcommand{\PreprocessorTok}[1]{\textcolor[rgb]{0.56,0.35,0.01}{\textit{#1}}}
\newcommand{\RegionMarkerTok}[1]{#1}
\newcommand{\SpecialCharTok}[1]{\textcolor[rgb]{0.00,0.00,0.00}{#1}}
\newcommand{\SpecialStringTok}[1]{\textcolor[rgb]{0.31,0.60,0.02}{#1}}
\newcommand{\StringTok}[1]{\textcolor[rgb]{0.31,0.60,0.02}{#1}}
\newcommand{\VariableTok}[1]{\textcolor[rgb]{0.00,0.00,0.00}{#1}}
\newcommand{\VerbatimStringTok}[1]{\textcolor[rgb]{0.31,0.60,0.02}{#1}}
\newcommand{\WarningTok}[1]{\textcolor[rgb]{0.56,0.35,0.01}{\textbf{\textit{#1}}}}
\usepackage{longtable,booktabs,array}
\usepackage{calc} % for calculating minipage widths
% Correct order of tables after \paragraph or \subparagraph
\usepackage{etoolbox}
\makeatletter
\patchcmd\longtable{\par}{\if@noskipsec\mbox{}\fi\par}{}{}
\makeatother
% Allow footnotes in longtable head/foot
\IfFileExists{footnotehyper.sty}{\usepackage{footnotehyper}}{\usepackage{footnote}}
\makesavenoteenv{longtable}
\usepackage{graphicx}
\makeatletter
\def\maxwidth{\ifdim\Gin@nat@width>\linewidth\linewidth\else\Gin@nat@width\fi}
\def\maxheight{\ifdim\Gin@nat@height>\textheight\textheight\else\Gin@nat@height\fi}
\makeatother
% Scale images if necessary, so that they will not overflow the page
% margins by default, and it is still possible to overwrite the defaults
% using explicit options in \includegraphics[width, height, ...]{}
\setkeys{Gin}{width=\maxwidth,height=\maxheight,keepaspectratio}
% Set default figure placement to htbp
\makeatletter
\def\fps@figure{htbp}
\makeatother
\setlength{\emergencystretch}{3em} % prevent overfull lines
\providecommand{\tightlist}{%
  \setlength{\itemsep}{0pt}\setlength{\parskip}{0pt}}
\setcounter{secnumdepth}{-\maxdimen} % remove section numbering
\usepackage{booktabs}
\usepackage{longtable}
\usepackage{array}
\usepackage{multirow}
\usepackage{wrapfig}
\usepackage{float}
\usepackage{colortbl}
\usepackage{pdflscape}
\usepackage{tabu}
\usepackage{threeparttable}
\usepackage{threeparttablex}
\usepackage[normalem]{ulem}
\usepackage{makecell}
\usepackage{xcolor}
\ifLuaTeX
  \usepackage{selnolig}  % disable illegal ligatures
\fi

\begin{document}
\maketitle

\hypertarget{project-final-report}{%
\subsection{Project Final Report}\label{project-final-report}}

\hypertarget{team-members}{%
\subsection{Team Members}\label{team-members}}

\begin{itemize}
\tightlist
\item
  Furkan Eskicioglu
\item
  Ugurcan Bayraktar
\end{itemize}

\includegraphics{https://blogs.agu.org/tremblingearth/files/2013/08/Seismogram.png}

\hypertarget{project-description}{%
\subsection{Project Description}\label{project-description}}

\hypertarget{project-goal-social-problem}{%
\subsubsection{Project Goal \& Social
Problem}\label{project-goal-social-problem}}

We have determined the earthquake, which is one of the natural disasters
that can be devastating and unpredictable, especially in regions that
are not used to earthquakes.

The aim of this project is to understand whether there is a relationship
between earthquakes in the world. In this direction, historical,
regional and trigger links between earthquakes were sought.

\hypertarget{project-data-access-to-data}{%
\subsubsection{Project data \& access to
data}\label{project-data-access-to-data}}

We knew that our dataset selection was important in order to make
earthquake data more meaningful, so we chose the United States
Geological Survey to access worldwide data, and Boğaziçi University
Kandilli Observatory and Earthquake Research Institute to access data
specific to Turkey. For this purpose, we used the earthquake data of the
USGS and KOERI for the years 2016-2020.

The datasets were easily obtained in the web interface thanks to the API
provided by the USGS and KOERI. The data used in the analysis consists
of data with a magnitude \textgreater2.5 in order to increase accuracy
and avoid confusion.

\hypertarget{actions-taken}{%
\subsection{Actions taken}\label{actions-taken}}

Within the scope of the project, we first tried to clean the data we
imported from the USGS and KOERI sites. Because they were included in
the dataset for uncertain earthquakes, we had to exclude them so that
they do not affect the analysis. When importing the data, it made our
job very easy as we got the size \textgreater2.5. In the next process,
we cleaned \textasciitilde2k lines of missing data. We reclassified the
variables by data types and looked at their statistics for numeric
variables to give us an idea. We then decided on the visualizations that
we thought might be useful and tried to draw them.

\hypertarget{introduction-to-datasets}{%
\paragraph{Introduction to Datasets}\label{introduction-to-datasets}}

\hypertarget{koeri-dataset}{%
\subparagraph{KOERI Dataset}\label{koeri-dataset}}

\href{http://www.koeri.boun.edu.tr/sismo/2/earthquake-catalog/}{\textbf{KOERI(Kandilli
Observatory and Earthquake Research Institute) Earthquake Catalog}} is
the website we will use for detailed Turkey earthquake data.

\hypertarget{the-parameters-and-their-explanations-in-this-data-are-given-below}{%
\paragraph{The parameters and their explanations in this data are given
below:}\label{the-parameters-and-their-explanations-in-this-data-are-given-below}}

\begin{longtable}[]{@{}
  >{\raggedright\arraybackslash}p{(\columnwidth - 2\tabcolsep) * \real{0.15}}
  >{\raggedright\arraybackslash}p{(\columnwidth - 2\tabcolsep) * \real{0.85}}@{}}
\toprule
\begin{minipage}[b]{\linewidth}\raggedright
Param Name
\end{minipage} & \begin{minipage}[b]{\linewidth}\raggedright
Description
\end{minipage} \\
\midrule
\endhead
No & Event Sequence \\
Event ID & Unic ID for event {[}YYYYMMDDHHMMSS
(YearMonthDayHourMinuteSecond){]} \\
Date & Date of event specified in the following format YYYY.MM.DD
(Year.Month.Day) \\
Origin Time & Origin time of event (UTC) specified in the following
format HH:MM:SS.MS \\
Latitude & in decimal degrees \\
Longitude & in decimal degrees \\
Depth(km) & Depth of the event in kilometers \\
Mag & Magnitude for the event \\
Type & Earthquake (Ke) or Suspected Explosion (Sm) \\
Location & Nearest settlement \\
\bottomrule
\end{longtable}

\hypertarget{usgs-dataset}{%
\subparagraph{USGS Dataset}\label{usgs-dataset}}

\href{https://earthquake.usgs.gov/earthquakes/search/}{\textbf{USGS
(United States Geological Survey) Search Catalog}} is the other website
we will use for the details of the world-wide earthquake data. We
obtained the datasets annually as seperate files, therefore these
datasets have to be merged.

\hypertarget{the-parameters-and-their-explanations-for-this-data-are-given-below}{%
\paragraph{The parameters and their explanations for this data are given
below:}\label{the-parameters-and-their-explanations-for-this-data-are-given-below}}

\begin{longtable}[]{@{}
  >{\raggedright\arraybackslash}p{(\columnwidth - 2\tabcolsep) * \real{0.15}}
  >{\raggedright\arraybackslash}p{(\columnwidth - 2\tabcolsep) * \real{0.85}}@{}}
\toprule
\begin{minipage}[b]{\linewidth}\raggedright
Param Name
\end{minipage} & \begin{minipage}[b]{\linewidth}\raggedright
Description
\end{minipage} \\
\midrule
\endhead
time & Time when the event occurred. Times are reported in milliseconds
since the epoch \\
latitude & Decimal degrees latitude. Negative values for southern
latitudes \\
longitude & Decimal degrees longitude. Negative values for western
longitudes \\
depth & Depth of the event in kilometers \\
mag & The magnitude for the event \\
magType & The method or algorithm used to calculate the preferred
magnitude for the event \\
nst & The total number of seismic stations used to determine earthquake
location \\
gap & The largest azimuthal gap between azimuthally adjacent stations
(in degrees) \\
dmin & Horizontal distance from the epicenter to the nearest station (in
degrees) \\
rms & The root-mean-square (RMS) travel time residual, in sec, using all
weights \\
net & The ID of a data contributor \\
id & A unique identifier for the event \\
updated & Time when the event was most recently updated \\
place & Textual description of named geographic region near to the
event \\
type & A comma-separated list of product types associated to this
event \\
horizontalError & Uncertainty of reported location of the event in
kilometers \\
depthError & Uncertainty of reported depth of the event in kilometers \\
magError & Uncertainty of reported magnitude of the event \\
magNst & The total number of seismic stations used to calculate the
magnitude for this earthquake \\
status & Indicates whether the event has been reviewed by a human \\
locationSource & The network that originally authored the reported
location of this event \\
magSource & Network that originally authored the reported magnitude for
this event \\
\bottomrule
\end{longtable}

\hypertarget{we-have-classified-earthquakes-according-to-their-magnitudes-2.5-4-4-5-5-6-6-7-7-9-formats-for-use-in-graphics.}{%
\paragraph{We have classified earthquakes according to their magnitudes
(2.5-4, 4-5, 5-6, 6-7, 7-9 formats) for use in
graphics.}\label{we-have-classified-earthquakes-according-to-their-magnitudes-2.5-4-4-5-5-6-6-7-7-9-formats-for-use-in-graphics.}}

\begin{Shaded}
\begin{Highlighting}[]
\CommentTok{\#KOERI}
\NormalTok{turkey\_tidyquake }\OtherTok{\textless{}{-}}\NormalTok{ turkey\_tidyquake }\SpecialCharTok{\%\textgreater{}\%} 
                      \FunctionTok{mutate}\NormalTok{(}\AttributeTok{magClass =} \FunctionTok{cut}\NormalTok{(Mag, }\AttributeTok{breaks=}\FunctionTok{c}\NormalTok{(}\FloatTok{2.4}\NormalTok{,}\DecValTok{4}\NormalTok{,}\DecValTok{5}\NormalTok{,}\DecValTok{6}\NormalTok{,}\DecValTok{7}\NormalTok{,}\DecValTok{9}\NormalTok{),}
                                            \AttributeTok{labels=}\FunctionTok{c}\NormalTok{(}\StringTok{"2.5{-}4"}\NormalTok{, }\StringTok{"4{-}5"}\NormalTok{, }\StringTok{"5{-}6"}\NormalTok{, }\StringTok{"6{-}7"}\NormalTok{, }\StringTok{"7{-}9"}\NormalTok{)))}
\CommentTok{\#USGS}
\NormalTok{data}\OtherTok{\textless{}{-}} \FunctionTok{mutate}\NormalTok{(data, }\AttributeTok{magClass=}\FunctionTok{cut}\NormalTok{(data}\SpecialCharTok{$}\NormalTok{mag, }\AttributeTok{breaks=}\FunctionTok{c}\NormalTok{(}\FloatTok{2.4}\NormalTok{, }\DecValTok{4}\NormalTok{, }\DecValTok{5}\NormalTok{, }\DecValTok{6}\NormalTok{, }\DecValTok{7}\NormalTok{, }\DecValTok{9}\NormalTok{), }\AttributeTok{labels=}\FunctionTok{c}\NormalTok{(}\StringTok{"2.5{-}4"}\NormalTok{, }\StringTok{"4{-}5"}\NormalTok{, }\StringTok{"5{-}6"}\NormalTok{, }\StringTok{"6{-}7"}\NormalTok{, }\StringTok{"7{-}9"}\NormalTok{)))}
\end{Highlighting}
\end{Shaded}

\hypertarget{quick-stats-summary-check-at-numeric-variables}{%
\paragraph{Quick stats summary check at numeric
variables}\label{quick-stats-summary-check-at-numeric-variables}}

\begin{Shaded}
\begin{Highlighting}[]
\FunctionTok{sapply}\NormalTok{(data[,}\FunctionTok{names}\NormalTok{(}\FunctionTok{which}\NormalTok{(}\FunctionTok{sapply}\NormalTok{(data, class) }\SpecialCharTok{==} \StringTok{"numeric"}\NormalTok{))],summary)}
\end{Highlighting}
\end{Shaded}

\begin{verbatim}
## $latitude
##     Min.  1st Qu.   Median     Mean  3rd Qu.     Max. 
## -82.8837   0.0914  19.4058  20.2736  42.3552  87.3860 
## 
## $longitude
##    Min. 1st Qu.  Median    Mean 3rd Qu.    Max. 
## -180.00 -155.26  -94.22  -49.60   74.00  180.00 
## 
## $depth
##    Min. 1st Qu.  Median    Mean 3rd Qu.    Max. 
##   -3.60    9.00   11.91   55.31   46.62  679.12 
## 
## $mag
##    Min. 1st Qu.  Median    Mean 3rd Qu.    Max. 
##   2.500   2.800   3.600   3.661   4.400   8.200 
## 
## $gap
##    Min. 1st Qu.  Median    Mean 3rd Qu.    Max.    NA's 
##     7.0    66.0   112.0   129.7   186.0   359.0   14300 
## 
## $dmin
##    Min. 1st Qu.  Median    Mean 3rd Qu.    Max.    NA's 
##   0.000   0.148   0.961   2.183   2.586 127.420   18523 
## 
## $rms
##    Min. 1st Qu.  Median    Mean 3rd Qu.    Max.    NA's 
##  0.0000  0.2800  0.5900  0.5932  0.8400 46.2400       5 
## 
## $horizontalError
##    Min. 1st Qu.  Median    Mean 3rd Qu.    Max.    NA's 
##   0.000   1.300   5.700   5.819   8.800  99.000   14206 
## 
## $depthError
##     Min.  1st Qu.   Median     Mean  3rd Qu.     Max.     NA's 
##    0.000    0.800    2.000    4.737    7.200 2329.800        7 
## 
## $magError
##    Min. 1st Qu.  Median    Mean 3rd Qu.    Max.    NA's 
##   0.000   0.078   0.129   0.249   0.200   5.670   22369
\end{verbatim}

\hypertarget{distribution-of-earthquakes-on-the-world-map-according-to-their-magnitude}{%
\paragraph{Distribution of earthquakes on the world map according to
their
magnitude}\label{distribution-of-earthquakes-on-the-world-map-according-to-their-magnitude}}

\begin{Shaded}
\begin{Highlighting}[]
\CommentTok{\#World map}
\NormalTok{p }\OtherTok{\textless{}{-}} \FunctionTok{ggplot}\NormalTok{() }\SpecialCharTok{+} \FunctionTok{geom\_map}\NormalTok{(}\AttributeTok{data =} \FunctionTok{map\_data}\NormalTok{(}\StringTok{"world"}\NormalTok{), }\AttributeTok{map =} \FunctionTok{map\_data}\NormalTok{(}\StringTok{"world"}\NormalTok{), }\FunctionTok{aes}\NormalTok{(}\AttributeTok{x =}\NormalTok{ long, }\AttributeTok{y=}\NormalTok{lat, }\AttributeTok{group=}\NormalTok{group, }\AttributeTok{map\_id=}\NormalTok{region), }\AttributeTok{fill=}\StringTok{"white"}\NormalTok{, }\AttributeTok{colour=}\StringTok{"\#7f7f7f"}\NormalTok{, }\AttributeTok{size=}\FloatTok{0.5}\NormalTok{)}

\CommentTok{\#Earthquakes points}
\NormalTok{p }\OtherTok{\textless{}{-}}\NormalTok{ p }\SpecialCharTok{+} \FunctionTok{geom\_point}\NormalTok{(}\AttributeTok{data =}\NormalTok{ data, }\FunctionTok{aes}\NormalTok{(}\AttributeTok{x=}\NormalTok{longitude, }\AttributeTok{y =}\NormalTok{ latitude, }\AttributeTok{colour =}\NormalTok{ mag)) }\SpecialCharTok{+} \FunctionTok{scale\_colour\_gradient}\NormalTok{(}\AttributeTok{low =} \StringTok{"\#00AA00"}\NormalTok{,}\AttributeTok{high =} \StringTok{"red"}\NormalTok{)}
\NormalTok{p}
\end{Highlighting}
\end{Shaded}

\includegraphics{project_presentation_NA_files/figure-latex/unnamed-chunk-13-1.pdf}

\hypertarget{distribution-of-earthquakes-on-turkey-according-to-their-magnitude}{%
\paragraph{Distribution of earthquakes on Turkey according to their
magnitude}\label{distribution-of-earthquakes-on-turkey-according-to-their-magnitude}}

Spatial map of Turkey is obtained from GADM(Database of Global
Administrative Areas)

\begin{Shaded}
\begin{Highlighting}[]
\CommentTok{\#url\_turkey \textless{}{-} "https://biogeo.ucdavis.edu/data/gadm3.6/Rsp/gadm36\_TUR\_1\_sp.rds"}
\CommentTok{\#download.file(url\_turkey, "data/sp\_turkey.rds")}

\NormalTok{tr\_sp }\OtherTok{\textless{}{-}} \FunctionTok{readRDS}\NormalTok{(}\StringTok{"data/sp\_turkey.rds"}\NormalTok{)}
\CommentTok{\#plot(tr\_sp)}

\CommentTok{\# Spatial data is converted to data frame below. Provinces are named below NAME\_1 column.}
\NormalTok{tr\_df }\OtherTok{\textless{}{-}}\NormalTok{ tr\_sp }\SpecialCharTok{\%\textgreater{}\%} 
          \FunctionTok{as\_tibble}\NormalTok{()}

\CommentTok{\# We obtain geographical data for Turkey.}
\NormalTok{tr\_geo }\OtherTok{\textless{}{-}}\NormalTok{ tr\_sp }\SpecialCharTok{\%\textgreater{}\%} 
            \FunctionTok{fortify}\NormalTok{()}

\CommentTok{\# Dataframe that contains provinces is created and joint to geographical data for Turkey. }
\FunctionTok{library}\NormalTok{(stringi) }\CommentTok{\#This is for Turkish characters such as ü, ş, etc.}
\NormalTok{prov }\OtherTok{=} \FunctionTok{tibble}\NormalTok{(}\AttributeTok{id =} \FunctionTok{rownames}\NormalTok{(tr\_df), }\AttributeTok{province =} \FunctionTok{stri\_trans\_general}\NormalTok{(tr\_df}\SpecialCharTok{$}\NormalTok{NAME\_1, }\StringTok{"Latin{-}ASCII"}\NormalTok{))}
\NormalTok{tr\_geo\_final }\OtherTok{\textless{}{-}} \FunctionTok{left\_join}\NormalTok{(tr\_geo, prov, }\AttributeTok{by =} \StringTok{"id"}\NormalTok{)}
\NormalTok{tr\_geo\_final}\SpecialCharTok{$}\NormalTok{id }\OtherTok{\textless{}{-}} \FunctionTok{as.numeric}\NormalTok{(tr\_geo\_final}\SpecialCharTok{$}\NormalTok{id)}

\CommentTok{\# Using our geographical data, we can draw the map of Turkey with ggplot.}
\NormalTok{tr\_map }\OtherTok{\textless{}{-}} \FunctionTok{ggplot}\NormalTok{()}\SpecialCharTok{+}
            \FunctionTok{theme\_minimal}\NormalTok{()}\SpecialCharTok{+}
            \CommentTok{\#Drawing the borders of Turkey and its provinces}
            \FunctionTok{geom\_polygon}\NormalTok{(}\AttributeTok{data =}\NormalTok{ tr\_geo\_final, }\FunctionTok{aes}\NormalTok{(}\AttributeTok{x=}\NormalTok{long, }\AttributeTok{y =}\NormalTok{ lat, }\AttributeTok{group =}\NormalTok{ group, }\AttributeTok{map\_id =}\NormalTok{ id),}
                         \AttributeTok{color =} \StringTok{"black"}\NormalTok{, }\AttributeTok{fill =} \StringTok{"white"}\NormalTok{) }\SpecialCharTok{+}
            \CommentTok{\#Fixed the scale of coordinate system}
            \FunctionTok{coord\_fixed}\NormalTok{()}

\CommentTok{\# At the end, the locations of the earthquakes are pointed on the map.}
\NormalTok{tr\_map }\SpecialCharTok{+}
  \FunctionTok{labs}\NormalTok{(}\AttributeTok{title =} \StringTok{"Earthquake Map of Turkey in Years 2016{-}2020"}\NormalTok{)}\SpecialCharTok{+}
  \FunctionTok{theme}\NormalTok{(}\AttributeTok{plot.title =} \FunctionTok{element\_text}\NormalTok{(}\AttributeTok{hjust =} \FloatTok{0.5}\NormalTok{)) }\SpecialCharTok{+}
  \FunctionTok{geom\_point}\NormalTok{(}\AttributeTok{data =}\NormalTok{ turkey\_tidyquake, }\FunctionTok{aes}\NormalTok{(}\AttributeTok{x =}\NormalTok{ Longitude, }\AttributeTok{y =}\NormalTok{ Latitude, }\AttributeTok{color =}\NormalTok{ Mag), }\AttributeTok{alpha=}\FloatTok{0.7}\NormalTok{) }\SpecialCharTok{+}
  \FunctionTok{scale\_color\_gradient}\NormalTok{(}\AttributeTok{low =} \StringTok{"\#00AA00"}\NormalTok{, }\AttributeTok{high =} \StringTok{"red"}\NormalTok{)}
\end{Highlighting}
\end{Shaded}

\includegraphics{project_presentation_NA_files/figure-latex/unnamed-chunk-14-1.pdf}

As seen above, earthquakes are occurred often on northwest to southwest
of Turkey. Also, we can see the earthquakes are common on The North
Anatolian Fault.

\hypertarget{distribution-of-the-number-of-earthquakes-by-years.}{%
\paragraph{Distribution of The Number of Earthquakes by
Years.}\label{distribution-of-the-number-of-earthquakes-by-years.}}

As seen below, more than 20 thousand earthquakes occurred each year from
2016 to 2020. In 2018, there were nearly twice as many earthquakes
occurred, compared to 2017. This is the highest count of earthquakes in
these 5 years.

\begin{Shaded}
\begin{Highlighting}[]
\NormalTok{year}\OtherTok{\textless{}{-}}\NormalTok{data }\SpecialCharTok{\%\textgreater{}\%} \FunctionTok{group\_by}\NormalTok{(year) }\SpecialCharTok{\%\textgreater{}\%} \FunctionTok{tally}\NormalTok{()}
\NormalTok{p }\OtherTok{\textless{}{-}} \FunctionTok{ggplot}\NormalTok{(year) }\SpecialCharTok{+} \FunctionTok{geom\_bar}\NormalTok{(}\FunctionTok{aes}\NormalTok{(}\AttributeTok{x=}\NormalTok{year, }\AttributeTok{y=}\NormalTok{n, }\AttributeTok{fill =} \FunctionTok{as.factor}\NormalTok{(n)), }\AttributeTok{stat=}\StringTok{"identity"}\NormalTok{)}\SpecialCharTok{+}
      \FunctionTok{scale\_fill\_brewer}\NormalTok{(}\AttributeTok{palette =} \StringTok{"YlOrRd"}\NormalTok{)}\SpecialCharTok{+}
      \FunctionTok{theme\_minimal}\NormalTok{()}\SpecialCharTok{+}
      \FunctionTok{theme}\NormalTok{(}\AttributeTok{legend.position =} \StringTok{"none"}\NormalTok{)}
\NormalTok{p }\OtherTok{\textless{}{-}}\NormalTok{ p }\SpecialCharTok{+} \FunctionTok{ggtitle}\NormalTok{(}\StringTok{"Distribution of Earthquake Counts in The World by Years"}\NormalTok{) }\SpecialCharTok{+}
      \FunctionTok{xlab}\NormalTok{(}\StringTok{"Years"}\NormalTok{) }\SpecialCharTok{+} \FunctionTok{ylab}\NormalTok{(}\StringTok{"Counts"}\NormalTok{)}

\NormalTok{p}
\end{Highlighting}
\end{Shaded}

\includegraphics{project_presentation_NA_files/figure-latex/unnamed-chunk-15-1.pdf}

In Turkey, there are more than one thousand earthquakes happened each
year from 2016 to 2020. In 2017, more than 5000 of earthquakes occurred
in Turkey, which is a peak in these 5 years.

\begin{Shaded}
\begin{Highlighting}[]
\NormalTok{year\_turkey }\OtherTok{\textless{}{-}}\NormalTok{ turkey\_tidyquake }\SpecialCharTok{\%\textgreater{}\%}
                \FunctionTok{group\_by}\NormalTok{(Year) }\SpecialCharTok{\%\textgreater{}\%} 
                \FunctionTok{tally}\NormalTok{()}
\NormalTok{tr\_yearthquake }\OtherTok{\textless{}{-}} \FunctionTok{ggplot}\NormalTok{(year\_turkey) }\SpecialCharTok{+} \FunctionTok{geom\_bar}\NormalTok{(}\FunctionTok{aes}\NormalTok{(}\AttributeTok{x =}\NormalTok{ Year, }\AttributeTok{y =}\NormalTok{ n, }\AttributeTok{fill =} \FunctionTok{as.factor}\NormalTok{(n)),}
                                                 \AttributeTok{stat =} \StringTok{"identity"}\NormalTok{)}\SpecialCharTok{+}
                  \FunctionTok{scale\_fill\_brewer}\NormalTok{(}\AttributeTok{palette =} \StringTok{"Blues"}\NormalTok{)}\SpecialCharTok{+}
                  \FunctionTok{theme\_minimal}\NormalTok{()}\SpecialCharTok{+}
                  \FunctionTok{theme}\NormalTok{(}\AttributeTok{legend.position =} \StringTok{"none"}\NormalTok{)}\SpecialCharTok{+}
                  \FunctionTok{ylab}\NormalTok{(}\StringTok{"Counts"}\NormalTok{)}
\NormalTok{tr\_yearthquake }\OtherTok{\textless{}{-}}\NormalTok{ tr\_yearthquake }\SpecialCharTok{+} \FunctionTok{ggtitle}\NormalTok{(}\StringTok{"Distribution of Earthquake Counts in Turkey by Years"}\NormalTok{)}
\NormalTok{tr\_yearthquake}
\end{Highlighting}
\end{Shaded}

\includegraphics{project_presentation_NA_files/figure-latex/unnamed-chunk-16-1.pdf}

\hypertarget{distribution-of-the-number-of-earthquakes-by-months.}{%
\paragraph{Distribution of The Number of Earthquakes by
Months.}\label{distribution-of-the-number-of-earthquakes-by-months.}}

There are more than 10 thousand earthquakes observed in the world each
year from 2016 to 2020. The count of earthquakes increased on Summer.
Therefore there could be a relationship between temperature and the
earthquakes.

\begin{Shaded}
\begin{Highlighting}[]
\FunctionTok{library}\NormalTok{(RColorBrewer)}
\NormalTok{month}\OtherTok{\textless{}{-}}\NormalTok{data }\SpecialCharTok{\%\textgreater{}\%} \FunctionTok{group\_by}\NormalTok{(month) }\SpecialCharTok{\%\textgreater{}\%} \FunctionTok{tally}\NormalTok{()}
\NormalTok{world\_monthquake }\OtherTok{\textless{}{-}} \FunctionTok{ggplot}\NormalTok{(month) }\SpecialCharTok{+} \FunctionTok{geom\_bar}\NormalTok{(}\FunctionTok{aes}\NormalTok{(}\AttributeTok{x=}\NormalTok{month, }\AttributeTok{y=}\NormalTok{n, }\AttributeTok{fill=} \FunctionTok{as.factor}\NormalTok{(n)), }\AttributeTok{stat=}\StringTok{"identity"}\NormalTok{) }\SpecialCharTok{+}
     \FunctionTok{scale\_fill\_manual}\NormalTok{(}\AttributeTok{values =} \FunctionTok{colorRampPalette}\NormalTok{(}\FunctionTok{brewer.pal}\NormalTok{(}\DecValTok{9}\NormalTok{,}\StringTok{"PuRd"}\NormalTok{))(}\DecValTok{12}\NormalTok{))}\SpecialCharTok{+}
     \FunctionTok{theme\_minimal}\NormalTok{()}\SpecialCharTok{+}
     \FunctionTok{theme}\NormalTok{(}\AttributeTok{legend.position =} \StringTok{"none"}\NormalTok{)}
\NormalTok{world\_monthquake }\OtherTok{\textless{}{-}}\NormalTok{ world\_monthquake }\SpecialCharTok{+} \FunctionTok{ggtitle}\NormalTok{(}\StringTok{"Distribution of Earthquake Counts in The World by Months"}\NormalTok{) }\SpecialCharTok{+} \FunctionTok{xlab}\NormalTok{(}\StringTok{"Months"}\NormalTok{) }\SpecialCharTok{+} \FunctionTok{ylab}\NormalTok{(}\StringTok{"Counts"}\NormalTok{)}
\NormalTok{world\_monthquake}
\end{Highlighting}
\end{Shaded}

\includegraphics{project_presentation_NA_files/figure-latex/unnamed-chunk-17-1.pdf}
In Turkey, we also see the number of earthquakes are increased on Summer
season, while the most of earthquakes occurred in January and February.

\begin{Shaded}
\begin{Highlighting}[]
\NormalTok{tr\_monthquake }\OtherTok{\textless{}{-}}\NormalTok{ turkey\_tidyquake }\SpecialCharTok{\%\textgreater{}\%} 
                  \FunctionTok{group\_by}\NormalTok{(Month) }\SpecialCharTok{\%\textgreater{}\%} 
                  \FunctionTok{tally}\NormalTok{()}
\NormalTok{tr\_month\_plot }\OtherTok{\textless{}{-}} \FunctionTok{ggplot}\NormalTok{(tr\_monthquake) }\SpecialCharTok{+} \FunctionTok{geom\_bar}\NormalTok{(}\FunctionTok{aes}\NormalTok{(}\AttributeTok{x=}\NormalTok{Month, }\AttributeTok{y=}\NormalTok{n, }\AttributeTok{fill=} \FunctionTok{as.factor}\NormalTok{(n)),}
                                                  \AttributeTok{stat =} \StringTok{"identity"}\NormalTok{) }\SpecialCharTok{+}
                  \FunctionTok{scale\_fill\_manual}\NormalTok{(}\AttributeTok{values =} \FunctionTok{colorRampPalette}\NormalTok{(}\FunctionTok{brewer.pal}\NormalTok{(}\DecValTok{9}\NormalTok{,}\StringTok{"Oranges"}\NormalTok{))(}\DecValTok{12}\NormalTok{))}\SpecialCharTok{+}
                  \FunctionTok{scale\_x\_continuous}\NormalTok{(}\AttributeTok{breaks =} \FunctionTok{c}\NormalTok{(}\DecValTok{1}\SpecialCharTok{:}\DecValTok{12}\NormalTok{))}\SpecialCharTok{+}
                  \FunctionTok{theme\_minimal}\NormalTok{()}\SpecialCharTok{+}
                  \FunctionTok{theme}\NormalTok{(}\AttributeTok{legend.position =} \StringTok{"none"}\NormalTok{)}
\NormalTok{tr\_month\_plot }\OtherTok{\textless{}{-}}\NormalTok{ tr\_month\_plot }\SpecialCharTok{+} 
                  \FunctionTok{ggtitle}\NormalTok{(}\StringTok{"Distribution of Earthquake Counts in Turkey by Months"}\NormalTok{)}
\NormalTok{tr\_month\_plot}
\end{Highlighting}
\end{Shaded}

\includegraphics{project_presentation_NA_files/figure-latex/unnamed-chunk-18-1.pdf}

\hypertarget{distribution-of-the-earthquakes-by-years-and-magnitude-classes.}{%
\paragraph{Distribution of The Earthquakes by years and magnitude
classes.}\label{distribution-of-the-earthquakes-by-years-and-magnitude-classes.}}

\begin{Shaded}
\begin{Highlighting}[]
\NormalTok{year}\OtherTok{\textless{}{-}}\NormalTok{data }\SpecialCharTok{\%\textgreater{}\%} \FunctionTok{group\_by}\NormalTok{(year, magClass) }\SpecialCharTok{\%\textgreater{}\%} \FunctionTok{tally}\NormalTok{()}
\NormalTok{year}\SpecialCharTok{\%\textgreater{}\%}\FunctionTok{ggplot}\NormalTok{(}\FunctionTok{aes}\NormalTok{(year, n))}\SpecialCharTok{+}
\FunctionTok{geom\_point}\NormalTok{(}\AttributeTok{size=}\DecValTok{1}\NormalTok{, }\AttributeTok{col=}\StringTok{"red"}\NormalTok{)}\SpecialCharTok{+}
  \FunctionTok{facet\_wrap}\NormalTok{(}\SpecialCharTok{\textasciitilde{}}\NormalTok{magClass,  }\AttributeTok{ncol=}\DecValTok{2}\NormalTok{, }\AttributeTok{scales=}\StringTok{"free"}\NormalTok{)}\SpecialCharTok{+}
   \FunctionTok{ggtitle}\NormalTok{(}\StringTok{"Number of Earthquakes by Magnitude Class and Year"}\NormalTok{) }\SpecialCharTok{+}
           \FunctionTok{xlab}\NormalTok{(}\StringTok{"Year"}\NormalTok{) }\SpecialCharTok{+} \FunctionTok{ylab}\NormalTok{(}\StringTok{"Number of Cases"}\NormalTok{)}\SpecialCharTok{+}
  \FunctionTok{theme}\NormalTok{(}\AttributeTok{plot.title =} \FunctionTok{element\_text}\NormalTok{(}\AttributeTok{face=}\StringTok{"bold"}\NormalTok{, }\AttributeTok{size=}\DecValTok{14}\NormalTok{, }\AttributeTok{hjust=}\FloatTok{0.5}\NormalTok{)) }\SpecialCharTok{+}
\FunctionTok{theme}\NormalTok{(}\AttributeTok{axis.title =} \FunctionTok{element\_text}\NormalTok{(}\AttributeTok{face=}\StringTok{"bold"}\NormalTok{, }\AttributeTok{size=}\DecValTok{12}\NormalTok{))}
\end{Highlighting}
\end{Shaded}

\includegraphics{project_presentation_NA_files/figure-latex/unnamed-chunk-19-1.pdf}

\hypertarget{lets-examine-the-distribution-of-earthquakes-on-earth-by-months-and-magnitude-classes.}{%
\paragraph{Let's examine the distribution of earthquakes on Earth by
months and magnitude
classes.}\label{lets-examine-the-distribution-of-earthquakes-on-earth-by-months-and-magnitude-classes.}}

\begin{Shaded}
\begin{Highlighting}[]
\NormalTok{month}\OtherTok{\textless{}{-}}\NormalTok{data }\SpecialCharTok{\%\textgreater{}\%} \FunctionTok{group\_by}\NormalTok{(month, magClass) }\SpecialCharTok{\%\textgreater{}\%} \FunctionTok{tally}\NormalTok{()}
\NormalTok{month}\SpecialCharTok{\%\textgreater{}\%}\FunctionTok{ggplot}\NormalTok{(}\FunctionTok{aes}\NormalTok{(month, n))}\SpecialCharTok{+}
\FunctionTok{geom\_point}\NormalTok{(}\AttributeTok{size=}\DecValTok{1}\NormalTok{, }\AttributeTok{col=}\StringTok{"red"}\NormalTok{)}\SpecialCharTok{+}
  \FunctionTok{facet\_wrap}\NormalTok{(}\SpecialCharTok{\textasciitilde{}}\NormalTok{magClass,  }\AttributeTok{ncol=}\DecValTok{2}\NormalTok{, }\AttributeTok{scales=}\StringTok{"free"}\NormalTok{)}\SpecialCharTok{+}
   \FunctionTok{ggtitle}\NormalTok{(}\StringTok{"Number of Earthquakes by Magnitude Class and Month"}\NormalTok{) }\SpecialCharTok{+}
           \FunctionTok{xlab}\NormalTok{(}\StringTok{"Months"}\NormalTok{) }\SpecialCharTok{+} \FunctionTok{ylab}\NormalTok{(}\StringTok{"Number of Cases"}\NormalTok{)}\SpecialCharTok{+}
  \FunctionTok{theme}\NormalTok{(}\AttributeTok{plot.title =} \FunctionTok{element\_text}\NormalTok{(}\AttributeTok{face=}\StringTok{"bold"}\NormalTok{, }\AttributeTok{size=}\DecValTok{14}\NormalTok{, }\AttributeTok{hjust=}\FloatTok{0.5}\NormalTok{)) }\SpecialCharTok{+}
\FunctionTok{theme}\NormalTok{(}\AttributeTok{axis.title =} \FunctionTok{element\_text}\NormalTok{(}\AttributeTok{face=}\StringTok{"bold"}\NormalTok{, }\AttributeTok{size=}\DecValTok{12}\NormalTok{))}
\end{Highlighting}
\end{Shaded}

\includegraphics{project_presentation_NA_files/figure-latex/unnamed-chunk-20-1.pdf}

\hypertarget{lets-check-the-density-function-of-magnitude.}{%
\paragraph{Let's check the density function of
Magnitude.}\label{lets-check-the-density-function-of-magnitude.}}

\begin{Shaded}
\begin{Highlighting}[]
\FunctionTok{ggdensity}\NormalTok{(data}\SpecialCharTok{$}\NormalTok{mag, }
          \AttributeTok{main =} \StringTok{"Density plot of magnitude"}\NormalTok{,}
          \AttributeTok{xlab =} \StringTok{"Magnitute"}\NormalTok{)}
\end{Highlighting}
\end{Shaded}

\includegraphics{project_presentation_NA_files/figure-latex/unnamed-chunk-21-1.pdf}

\hypertarget{results-and-discussion}{%
\subsection{Results and Discussion}\label{results-and-discussion}}

As a result of our research, we have reached: We learned the averages of
earthquake magnitudes in the world. On the world map, we have seen that
earthquakes are more intense on coastlines. Compared to other years, we
saw that the number of earthquakes in 2018 was almost doubled. We
observed an increase in the number of earthquakes in the summer months.
This increase seems especially 2.5-4 magnitude range.

Thanks to our project, we obtained separate insights from earthquake
data of the World and Turkey. We checked whether the earthquake
relations are a link between the world and Turkey. We compared the world
averages with Turkey, which is known as the earthquake zone. In this
process, we determined that we could make comparisons on the basis of
size, hourly, seasonal, seasonal (sea/terrestrial) and we focused on
these factors in our research accordingly.

\hypertarget{conclusion}{%
\subsection{Conclusion}\label{conclusion}}

As a result, we used data transfer, cleaning, reconstruction according
to data types and basic visualization processes in the analysis of
earthquake data in the world and Turkey between the years 2016-2020,
which we obtained from USGS and KOERI organizations, which we identified
as reliable data sources. In this way, we provided the opportunity to
visually see whether there is a similarity between the earthquakes that
took place in the world and in Turkey. In this way, we tried to figure
out whether our country, which we refer to as an earthquake zone, is
really an above-average earthquake zone when compared to other countries
in the world.

You can also access our project's GitHub page here:
\href{https://github.com/MAT381E-Fall21/project_final_report-na}{Statistics
of earthquake hazards in Turkey and comparison with the world}

\hypertarget{references}{%
\subsection{References}\label{references}}

\begin{itemize}
\tightlist
\item
  \url{https://earthquake.usgs.gov/earthquakes/search/}
\item
  \url{https://earthquake.usgs.gov/data/comcat/index.php}
\item
  \url{http://www.koeri.boun.edu.tr/sismo/2/earthquake-catalog/}
\item
  \url{https://towardsdatascience.com/reverse-geocoding-in-r-f7fe4b908355}
\item
  \url{https://tevfikbulut.com/2020/02/02/exploratory-data-analysis-of-turkey-earthquakes-ii/}
\item
  \url{https://www.r-bloggers.com/2019/04/earthquake-analysis-1-4-quantitative-variables-exploratory-analysis/}
\end{itemize}

\end{document}
