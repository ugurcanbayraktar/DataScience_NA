% Options for packages loaded elsewhere
\PassOptionsToPackage{unicode}{hyperref}
\PassOptionsToPackage{hyphens}{url}
%
\documentclass[
]{article}
\title{Statistics of earthquake hazards in Turkey and comparison with
the world}
\author{Team NA}
\date{1/30/2022}

\usepackage{amsmath,amssymb}
\usepackage{lmodern}
\usepackage{iftex}
\ifPDFTeX
  \usepackage[T1]{fontenc}
  \usepackage[utf8]{inputenc}
  \usepackage{textcomp} % provide euro and other symbols
\else % if luatex or xetex
  \usepackage{unicode-math}
  \defaultfontfeatures{Scale=MatchLowercase}
  \defaultfontfeatures[\rmfamily]{Ligatures=TeX,Scale=1}
\fi
% Use upquote if available, for straight quotes in verbatim environments
\IfFileExists{upquote.sty}{\usepackage{upquote}}{}
\IfFileExists{microtype.sty}{% use microtype if available
  \usepackage[]{microtype}
  \UseMicrotypeSet[protrusion]{basicmath} % disable protrusion for tt fonts
}{}
\makeatletter
\@ifundefined{KOMAClassName}{% if non-KOMA class
  \IfFileExists{parskip.sty}{%
    \usepackage{parskip}
  }{% else
    \setlength{\parindent}{0pt}
    \setlength{\parskip}{6pt plus 2pt minus 1pt}}
}{% if KOMA class
  \KOMAoptions{parskip=half}}
\makeatother
\usepackage{xcolor}
\IfFileExists{xurl.sty}{\usepackage{xurl}}{} % add URL line breaks if available
\IfFileExists{bookmark.sty}{\usepackage{bookmark}}{\usepackage{hyperref}}
\hypersetup{
  pdftitle={Statistics of earthquake hazards in Turkey and comparison with the world},
  pdfauthor={Team NA},
  hidelinks,
  pdfcreator={LaTeX via pandoc}}
\urlstyle{same} % disable monospaced font for URLs
\usepackage[margin=1in]{geometry}
\usepackage{color}
\usepackage{fancyvrb}
\newcommand{\VerbBar}{|}
\newcommand{\VERB}{\Verb[commandchars=\\\{\}]}
\DefineVerbatimEnvironment{Highlighting}{Verbatim}{commandchars=\\\{\}}
% Add ',fontsize=\small' for more characters per line
\usepackage{framed}
\definecolor{shadecolor}{RGB}{248,248,248}
\newenvironment{Shaded}{\begin{snugshade}}{\end{snugshade}}
\newcommand{\AlertTok}[1]{\textcolor[rgb]{0.94,0.16,0.16}{#1}}
\newcommand{\AnnotationTok}[1]{\textcolor[rgb]{0.56,0.35,0.01}{\textbf{\textit{#1}}}}
\newcommand{\AttributeTok}[1]{\textcolor[rgb]{0.77,0.63,0.00}{#1}}
\newcommand{\BaseNTok}[1]{\textcolor[rgb]{0.00,0.00,0.81}{#1}}
\newcommand{\BuiltInTok}[1]{#1}
\newcommand{\CharTok}[1]{\textcolor[rgb]{0.31,0.60,0.02}{#1}}
\newcommand{\CommentTok}[1]{\textcolor[rgb]{0.56,0.35,0.01}{\textit{#1}}}
\newcommand{\CommentVarTok}[1]{\textcolor[rgb]{0.56,0.35,0.01}{\textbf{\textit{#1}}}}
\newcommand{\ConstantTok}[1]{\textcolor[rgb]{0.00,0.00,0.00}{#1}}
\newcommand{\ControlFlowTok}[1]{\textcolor[rgb]{0.13,0.29,0.53}{\textbf{#1}}}
\newcommand{\DataTypeTok}[1]{\textcolor[rgb]{0.13,0.29,0.53}{#1}}
\newcommand{\DecValTok}[1]{\textcolor[rgb]{0.00,0.00,0.81}{#1}}
\newcommand{\DocumentationTok}[1]{\textcolor[rgb]{0.56,0.35,0.01}{\textbf{\textit{#1}}}}
\newcommand{\ErrorTok}[1]{\textcolor[rgb]{0.64,0.00,0.00}{\textbf{#1}}}
\newcommand{\ExtensionTok}[1]{#1}
\newcommand{\FloatTok}[1]{\textcolor[rgb]{0.00,0.00,0.81}{#1}}
\newcommand{\FunctionTok}[1]{\textcolor[rgb]{0.00,0.00,0.00}{#1}}
\newcommand{\ImportTok}[1]{#1}
\newcommand{\InformationTok}[1]{\textcolor[rgb]{0.56,0.35,0.01}{\textbf{\textit{#1}}}}
\newcommand{\KeywordTok}[1]{\textcolor[rgb]{0.13,0.29,0.53}{\textbf{#1}}}
\newcommand{\NormalTok}[1]{#1}
\newcommand{\OperatorTok}[1]{\textcolor[rgb]{0.81,0.36,0.00}{\textbf{#1}}}
\newcommand{\OtherTok}[1]{\textcolor[rgb]{0.56,0.35,0.01}{#1}}
\newcommand{\PreprocessorTok}[1]{\textcolor[rgb]{0.56,0.35,0.01}{\textit{#1}}}
\newcommand{\RegionMarkerTok}[1]{#1}
\newcommand{\SpecialCharTok}[1]{\textcolor[rgb]{0.00,0.00,0.00}{#1}}
\newcommand{\SpecialStringTok}[1]{\textcolor[rgb]{0.31,0.60,0.02}{#1}}
\newcommand{\StringTok}[1]{\textcolor[rgb]{0.31,0.60,0.02}{#1}}
\newcommand{\VariableTok}[1]{\textcolor[rgb]{0.00,0.00,0.00}{#1}}
\newcommand{\VerbatimStringTok}[1]{\textcolor[rgb]{0.31,0.60,0.02}{#1}}
\newcommand{\WarningTok}[1]{\textcolor[rgb]{0.56,0.35,0.01}{\textbf{\textit{#1}}}}
\usepackage{longtable,booktabs,array}
\usepackage{calc} % for calculating minipage widths
% Correct order of tables after \paragraph or \subparagraph
\usepackage{etoolbox}
\makeatletter
\patchcmd\longtable{\par}{\if@noskipsec\mbox{}\fi\par}{}{}
\makeatother
% Allow footnotes in longtable head/foot
\IfFileExists{footnotehyper.sty}{\usepackage{footnotehyper}}{\usepackage{footnote}}
\makesavenoteenv{longtable}
\usepackage{graphicx}
\makeatletter
\def\maxwidth{\ifdim\Gin@nat@width>\linewidth\linewidth\else\Gin@nat@width\fi}
\def\maxheight{\ifdim\Gin@nat@height>\textheight\textheight\else\Gin@nat@height\fi}
\makeatother
% Scale images if necessary, so that they will not overflow the page
% margins by default, and it is still possible to overwrite the defaults
% using explicit options in \includegraphics[width, height, ...]{}
\setkeys{Gin}{width=\maxwidth,height=\maxheight,keepaspectratio}
% Set default figure placement to htbp
\makeatletter
\def\fps@figure{htbp}
\makeatother
\setlength{\emergencystretch}{3em} % prevent overfull lines
\providecommand{\tightlist}{%
  \setlength{\itemsep}{0pt}\setlength{\parskip}{0pt}}
\setcounter{secnumdepth}{-\maxdimen} % remove section numbering
\usepackage{booktabs}
\usepackage{longtable}
\usepackage{array}
\usepackage{multirow}
\usepackage{wrapfig}
\usepackage{float}
\usepackage{colortbl}
\usepackage{pdflscape}
\usepackage{tabu}
\usepackage{threeparttable}
\usepackage{threeparttablex}
\usepackage[normalem]{ulem}
\usepackage{makecell}
\usepackage{xcolor}
\ifLuaTeX
  \usepackage{selnolig}  % disable illegal ligatures
\fi

\begin{document}
\maketitle

\hypertarget{project-final-report}{%
\subsection{Project Final Report}\label{project-final-report}}

\hypertarget{team-members}{%
\subsection{Team Members}\label{team-members}}

\begin{itemize}
\tightlist
\item
  Furkan Eskicioglu
\item
  Ugurcan Bayraktar
\end{itemize}

\includegraphics{https://blogs.agu.org/tremblingearth/files/2013/08/Seismogram.png}

\hypertarget{project-description}{%
\subsection{Project Description}\label{project-description}}

\hypertarget{project-goal-social-problem}{%
\subsubsection{Project Goal \& Social
Problem}\label{project-goal-social-problem}}

We have determined the earthquake, which is one of the natural disasters
that can be devastating and unpredictable, especially in regions that
are not used to earthquakes.

The aim of this project is to understand whether there is a relationship
between earthquakes in the world. In this direction, historical,
regional and trigger links between earthquakes were sought.

\hypertarget{project-data-access-to-data}{%
\subsubsection{Project data \& access to
data}\label{project-data-access-to-data}}

We knew that our dataset selection was important in order to make
earthquake data more meaningful, so we chose the United States
Geological Survey to access worldwide data, and Boğaziçi University
Kandilli Observatory and Earthquake Research Institute to access data
specific to Turkey. For this purpose, we used the earthquake data of the
USGS and KOERI for the years 2016-2020.

The datasets were easily obtained in the web interface thanks to the API
provided by the USGS and KOERI. The data used in the analysis consists
of data with a magnitude \textgreater2.5 in order to increase accuracy
and avoid confusion.

\hypertarget{actions-taken}{%
\subsection{Actions taken}\label{actions-taken}}

\hypertarget{warning-in-this-section-weve-included-all-the-code-blocks-we-used-to-be-self-explanatory.-in-the-oral-presentation-file-you-can-only-see-the-document-with-insights}{%
\subsubsection{Warning: In this section, we've included all the code
blocks we used to be self-explanatory. In the oral presentation file,
you can only see the document with
insights}\label{warning-in-this-section-weve-included-all-the-code-blocks-we-used-to-be-self-explanatory.-in-the-oral-presentation-file-you-can-only-see-the-document-with-insights}}

Within the scope of the project, we first tried to clean the data we
imported from the USGS and KOERI sites. Because they were included in
the dataset for uncertain earthquakes, we had to exclude them so that
they do not affect the analysis. When importing the data, it made our
job very easy as we got the size \textgreater2.5. In the next process,
we cleaned \textasciitilde2k lines of missing data. We reclassified the
variables by data types and looked at their statistics for numeric
variables to give us an idea. We then decided on the visualizations that
we thought might be useful and tried to draw them.

\hypertarget{install-libraries}{%
\paragraph{Install libraries}\label{install-libraries}}

\begin{Shaded}
\begin{Highlighting}[]
\CommentTok{\#Prerequisites}
\FunctionTok{install.packages}\NormalTok{(}\StringTok{"maps"}\NormalTok{)}
\FunctionTok{install.packages}\NormalTok{(}\StringTok{"ggpubr"}\NormalTok{)}
\FunctionTok{install.packages}\NormalTok{(}\StringTok{"kableExtra"}\NormalTok{)}
\end{Highlighting}
\end{Shaded}

\hypertarget{loading-libraries}{%
\subsubsection{Loading libraries}\label{loading-libraries}}

\begin{Shaded}
\begin{Highlighting}[]
\FunctionTok{library}\NormalTok{(tidyverse)}\CommentTok{\#for data manipulation}
\FunctionTok{library}\NormalTok{(lubridate)}\CommentTok{\#for formatting date and time}
\FunctionTok{library}\NormalTok{(kableExtra)}\CommentTok{\#for printing tables}
\FunctionTok{library}\NormalTok{(readxl)}\CommentTok{\#for reading excel file}
\FunctionTok{library}\NormalTok{(ggplot2) }\CommentTok{\#for graphs}
\FunctionTok{library}\NormalTok{(maps) }\CommentTok{\#for world map}
\FunctionTok{library}\NormalTok{(sp) }\CommentTok{\#for spatial data}
\FunctionTok{library}\NormalTok{(ggpubr)}\CommentTok{\#for density function}
\end{Highlighting}
\end{Shaded}

\hypertarget{loading-datasets}{%
\paragraph{Loading datasets}\label{loading-datasets}}

Dataset for the earthquake occured in Turkey has been obtained from
Kandilli Observatory and Earthquake Research Institute (KOERI)
\href{http://www.koeri.boun.edu.tr/sismo/zeqdb/}{\textbf{Database
Search}}. Data is retrieved as txt format, then pasted to an excel(xlsx)
file.

\href{https://earthquake.usgs.gov/earthquakes/search/}{\textbf{USGS
(United States Geological Survey) Search Catalog}} is the other website
we will use for the details of the world-wide earthquake data.

\begin{Shaded}
\begin{Highlighting}[]
\CommentTok{\#KOERI dataset}
\NormalTok{turkey\_earthquake }\OtherTok{\textless{}{-}} \FunctionTok{read\_excel}\NormalTok{(}\StringTok{"data/boun.xlsx"}\NormalTok{)}

\CommentTok{\#USGS datasets}
\NormalTok{data2016\_1 }\OtherTok{\textless{}{-}} \FunctionTok{read.csv}\NormalTok{(}\StringTok{"data/query 2016{-}1.csv"}\NormalTok{)}
\NormalTok{data2016\_2 }\OtherTok{\textless{}{-}} \FunctionTok{read.csv}\NormalTok{(}\StringTok{"data/query 2016{-}2.csv"}\NormalTok{)}
\NormalTok{data2017\_1 }\OtherTok{\textless{}{-}} \FunctionTok{read.csv}\NormalTok{(}\StringTok{"data/query 2017{-}1.csv"}\NormalTok{)}
\NormalTok{data2017\_2 }\OtherTok{\textless{}{-}} \FunctionTok{read.csv}\NormalTok{(}\StringTok{"data/query 2017{-}2.csv"}\NormalTok{)}
\NormalTok{data2018\_1 }\OtherTok{\textless{}{-}} \FunctionTok{read.csv}\NormalTok{(}\StringTok{"data/query 2018{-}1.csv"}\NormalTok{)}
\NormalTok{data2018\_2 }\OtherTok{\textless{}{-}} \FunctionTok{read.csv}\NormalTok{(}\StringTok{"data/query 2018{-}2.csv"}\NormalTok{)}
\NormalTok{data2018\_3 }\OtherTok{\textless{}{-}} \FunctionTok{read.csv}\NormalTok{(}\StringTok{"data/query 2018{-}3.csv"}\NormalTok{)}
\NormalTok{data2019\_1 }\OtherTok{\textless{}{-}} \FunctionTok{read.csv}\NormalTok{(}\StringTok{"data/query 2019{-}1.csv"}\NormalTok{)}
\NormalTok{data2019\_2 }\OtherTok{\textless{}{-}} \FunctionTok{read.csv}\NormalTok{(}\StringTok{"data/query 2019{-}2.csv"}\NormalTok{)}
\NormalTok{data2020\_1 }\OtherTok{\textless{}{-}} \FunctionTok{read.csv}\NormalTok{(}\StringTok{"data/query 2020{-}1.csv"}\NormalTok{)}
\NormalTok{data2020\_2 }\OtherTok{\textless{}{-}} \FunctionTok{read.csv}\NormalTok{(}\StringTok{"data/query 2020{-}2.csv"}\NormalTok{)}
\end{Highlighting}
\end{Shaded}

\hypertarget{introduction-to-datasets}{%
\paragraph{Introduction to Datasets}\label{introduction-to-datasets}}

\hypertarget{koeri-dataset}{%
\subparagraph{KOERI Dataset}\label{koeri-dataset}}

\href{http://www.koeri.boun.edu.tr/sismo/2/earthquake-catalog/}{\textbf{KOERI(Kandilli
Observatory and Earthquake Research Institute) Earthquake Catalog}} is
the website we will use for detailed Turkey earthquake data.

\begin{Shaded}
\begin{Highlighting}[]
\NormalTok{turkey\_tidyquake }\OtherTok{\textless{}{-}}\NormalTok{ turkey\_earthquake }\SpecialCharTok{\%\textgreater{}\%} 
                      \FunctionTok{select}\NormalTok{(No,}
                             \AttributeTok{Event\_ID =} \StringTok{\textasciigrave{}}\AttributeTok{Deprem Kodu}\StringTok{\textasciigrave{}}\NormalTok{,}
                             \AttributeTok{Date =} \StringTok{\textasciigrave{}}\AttributeTok{Olus tarihi}\StringTok{\textasciigrave{}}\NormalTok{,}
                             \AttributeTok{Origin\_Time =} \StringTok{\textasciigrave{}}\AttributeTok{Olus zamani}\StringTok{\textasciigrave{}}\NormalTok{,}
                             \AttributeTok{Latitude =}\NormalTok{ Enlem,}
                             \AttributeTok{Longitude =}\NormalTok{ Boylam,}
                             \AttributeTok{Depth\_km =} \StringTok{\textasciigrave{}}\AttributeTok{Der(km)}\StringTok{\textasciigrave{}}\NormalTok{,}
                             \AttributeTok{Mag =}\NormalTok{ xM,}
                             \AttributeTok{Type =}\NormalTok{ Tip) }\SpecialCharTok{\%\textgreater{}\%} 
                      \FunctionTok{filter}\NormalTok{(Type }\SpecialCharTok{==} \StringTok{"Ke"}\NormalTok{)}
\FunctionTok{head}\NormalTok{(turkey\_tidyquake)}
\end{Highlighting}
\end{Shaded}

\begin{verbatim}
## # A tibble: 6 x 9
##      No Event_ID Date      Origin_Time         Latitude Longitude Depth_km   Mag
##   <dbl>    <dbl> <chr>     <dttm>                 <dbl>     <dbl>    <dbl> <dbl>
## 1     1  2.02e13 2020.07.~ 1899-12-31 16:28:07     38.5      27.5     16.5   2.5
## 2     2  2.02e13 2020.07.~ 1899-12-31 12:55:40     38.5      27.5      8.6   3  
## 3     3  2.02e13 2020.07.~ 1899-12-31 12:45:17     36.7      28.2     67.1   2.5
## 4     4  2.02e13 2020.07.~ 1899-12-31 12:25:11     38.5      27.5      3.8   2.6
## 5     5  2.02e13 2020.07.~ 1899-12-31 10:44:28     38.5      27.5      8.6   3  
## 6     6  2.02e13 2020.07.~ 1899-12-31 09:59:45     38.5      27.5     13.3   3.8
## # ... with 1 more variable: Type <chr>
\end{verbatim}

\hypertarget{the-parameters-and-their-explanations-in-this-data-are-given-below}{%
\paragraph{The parameters and their explanations in this data are given
below:}\label{the-parameters-and-their-explanations-in-this-data-are-given-below}}

\begin{longtable}[]{@{}
  >{\raggedright\arraybackslash}p{(\columnwidth - 2\tabcolsep) * \real{0.15}}
  >{\raggedright\arraybackslash}p{(\columnwidth - 2\tabcolsep) * \real{0.85}}@{}}
\toprule
\begin{minipage}[b]{\linewidth}\raggedright
Param Name
\end{minipage} & \begin{minipage}[b]{\linewidth}\raggedright
Description
\end{minipage} \\
\midrule
\endhead
No & Event Sequence \\
Event ID & Unic ID for event {[}YYYYMMDDHHMMSS
(YearMonthDayHourMinuteSecond){]} \\
Date & Date of event specified in the following format YYYY.MM.DD
(Year.Month.Day) \\
Origin Time & Origin time of event (UTC) specified in the following
format HH:MM:SS.MS \\
Latitude & in decimal degrees \\
Longitude & in decimal degrees \\
Depth(km) & Depth of the event in kilometers \\
Mag & Magnitude for the event \\
Type & Earthquake (Ke) or Suspected Explosion (Sm) \\
Location & Nearest settlement \\
\bottomrule
\end{longtable}

\hypertarget{usgs-dataset}{%
\subparagraph{USGS Dataset}\label{usgs-dataset}}

\href{https://earthquake.usgs.gov/earthquakes/search/}{\textbf{USGS
(United States Geological Survey) Search Catalog}} is the other website
we will use for the details of the world-wide earthquake data. We
obtained the datasets annually as seperate files, therefore these
datasets have to be merged.

\begin{Shaded}
\begin{Highlighting}[]
\NormalTok{rawData }\OtherTok{\textless{}{-}} \FunctionTok{rbind}\NormalTok{(data2016\_1, }
\NormalTok{                data2016\_2, }
\NormalTok{                data2017\_1, }
\NormalTok{                data2017\_2, }
\NormalTok{                data2018\_1, }
\NormalTok{                data2018\_2, }
\NormalTok{                data2018\_3, }
\NormalTok{                data2019\_1, }
\NormalTok{                data2019\_2, }
\NormalTok{                data2020\_1, }
\NormalTok{                data2020\_2)}
\FunctionTok{head}\NormalTok{(rawData)}
\end{Highlighting}
\end{Shaded}

\begin{verbatim}
##                       time latitude longitude  depth mag magType nst   gap
## 1 2016-06-30T23:35:00.100Z  17.9638  -68.5780 65.000 3.8      Md  18 194.4
## 2 2016-06-30T23:04:59.230Z  36.7601  137.9208 10.300 4.5      mb  NA  42.0
## 3 2016-06-30T22:51:19.680Z  16.7209  146.3311 74.260 4.4      mb  NA  77.0
## 4 2016-06-30T22:25:03.700Z  36.4771  -98.7412  7.694 3.5     mwr  NA  51.0
## 5 2016-06-30T22:20:10.570Z  17.3387  147.4923 54.440 4.3      mb  NA 171.0
## 6 2016-06-30T21:47:49.290Z  -2.6333  139.0413 51.200 4.1      mb  NA  87.0
##       dmin  rms net         id                  updated
## 1 1.118403 0.56  pr pr16182026 2016-08-31T03:08:24.040Z
## 2 0.312000 0.56  us us10005yyt 2016-08-31T03:08:24.040Z
## 3 1.561000 0.81  us us1000619a 2016-08-31T03:08:24.040Z
## 4       NA 0.25  us us10005yy6 2016-08-31T03:08:24.040Z
## 5 2.650000 0.82  us us1000619h 2016-08-31T03:08:24.040Z
## 6 6.790000 0.72  us us1000619b 2016-08-31T03:08:24.040Z
##                                            place       type horizontalError
## 1    45 km S of Boca de Yuma, Dominican Republic earthquake             2.7
## 2                       8 km NE of Hakuba, Japan earthquake             3.3
## 3 177 km NNE of Saipan, Northern Mariana Islands earthquake             8.8
## 4                  17 km SE of Waynoka, Oklahoma earthquake             1.2
## 5  299 km NE of Saipan, Northern Mariana Islands earthquake            10.1
## 6                 176 km W of Abepura, Indonesia earthquake             6.6
##   depthError magError magNst   status locationSource magSource
## 1        7.1    0.000     15 reviewed             pr        pr
## 2        4.1    0.045    147 reviewed             us        us
## 3        6.8    0.096     31 reviewed             us        us
## 4        1.7       NA     12 reviewed            tul       slm
## 5        9.5    0.153     12 reviewed             us        us
## 6        8.7    0.130     16 reviewed             us        us
\end{verbatim}

\hypertarget{removing-useless-columns-and-saving-the-all_data}{%
\paragraph{Removing useless columns and saving the
all\_data}\label{removing-useless-columns-and-saving-the-all_data}}

\begin{Shaded}
\begin{Highlighting}[]
\NormalTok{data }\OtherTok{\textless{}{-}} \FunctionTok{select}\NormalTok{(rawData, }\SpecialCharTok{{-}}\FunctionTok{c}\NormalTok{(}\StringTok{"nst"}\NormalTok{,}\StringTok{"id"}\NormalTok{,}\StringTok{"updated"}\NormalTok{))}

\FunctionTok{write.csv}\NormalTok{(data,}\StringTok{"data/all\_data.csv"}\NormalTok{)}
\end{Highlighting}
\end{Shaded}

\hypertarget{data-structures}{%
\paragraph{Data Structures}\label{data-structures}}

\hypertarget{koeri}{%
\subparagraph{KOERI}\label{koeri}}

\begin{Shaded}
\begin{Highlighting}[]
\FunctionTok{str}\NormalTok{(turkey\_tidyquake)}
\end{Highlighting}
\end{Shaded}

\begin{verbatim}
## tibble [14,517 x 9] (S3: tbl_df/tbl/data.frame)
##  $ No         : num [1:14517] 1 2 3 4 5 6 7 8 9 10 ...
##  $ Event_ID   : num [1:14517] 2.02e+13 2.02e+13 2.02e+13 2.02e+13 2.02e+13 ...
##  $ Date       : chr [1:14517] "2020.07.01" "2020.07.01" "2020.07.01" "2020.07.01" ...
##  $ Origin_Time: POSIXct[1:14517], format: "1899-12-31 16:28:07" "1899-12-31 12:55:40" ...
##  $ Latitude   : num [1:14517] 38.5 38.5 36.7 38.5 38.5 ...
##  $ Longitude  : num [1:14517] 27.5 27.5 28.2 27.5 27.5 ...
##  $ Depth_km   : num [1:14517] 16.5 8.6 67.1 3.8 8.6 13.3 5 15.4 12.8 10.6 ...
##  $ Mag        : num [1:14517] 2.5 3 2.5 2.6 3 3.8 2.6 2.5 2.8 2.5 ...
##  $ Type       : chr [1:14517] "Ke" "Ke" "Ke" "Ke" ...
\end{verbatim}

As seen above, date is in character class. It is convenient to convert
it to date class for our data manipulations.

\begin{Shaded}
\begin{Highlighting}[]
\NormalTok{turkey\_tidyquake}\SpecialCharTok{$}\NormalTok{Date }\OtherTok{\textless{}{-}} \FunctionTok{as.Date}\NormalTok{(turkey\_tidyquake}\SpecialCharTok{$}\NormalTok{Date, }\AttributeTok{format =} \StringTok{"\%Y.\%m.\%d"}\NormalTok{)}
\FunctionTok{str}\NormalTok{(turkey\_tidyquake}\SpecialCharTok{$}\NormalTok{Date)}
\end{Highlighting}
\end{Shaded}

\begin{verbatim}
##  Date[1:14517], format: "2020-07-01" "2020-07-01" "2020-07-01" "2020-07-01" "2020-07-01" ...
\end{verbatim}

\begin{Shaded}
\begin{Highlighting}[]
\CommentTok{\#Reading all data from one source}
\NormalTok{data }\OtherTok{\textless{}{-}} \FunctionTok{read.csv}\NormalTok{(}\StringTok{"data/all\_data.csv"}\NormalTok{)}
\FunctionTok{str}\NormalTok{(data)}
\end{Highlighting}
\end{Shaded}

\begin{verbatim}
## 'data.frame':    152012 obs. of  20 variables:
##  $ X              : int  1 2 3 4 5 6 7 8 9 10 ...
##  $ time           : chr  "2016-06-30T23:35:00.100Z" "2016-06-30T23:04:59.230Z" "2016-06-30T22:51:19.680Z" "2016-06-30T22:25:03.700Z" ...
##  $ latitude       : num  18 36.8 16.7 36.5 17.3 ...
##  $ longitude      : num  -68.6 137.9 146.3 -98.7 147.5 ...
##  $ depth          : num  65 10.3 74.26 7.69 54.44 ...
##  $ mag            : num  3.8 4.5 4.4 3.5 4.3 4.1 4.3 4.5 2.6 4.9 ...
##  $ magType        : chr  "Md" "mb" "mb" "mwr" ...
##  $ gap            : num  194 42 77 51 171 ...
##  $ dmin           : num  1.118 0.312 1.561 NA 2.65 ...
##  $ rms            : num  0.56 0.56 0.81 0.25 0.82 0.72 1.27 1.02 0.58 1.22 ...
##  $ net            : chr  "pr" "us" "us" "us" ...
##  $ place          : chr  "45 km S of Boca de Yuma, Dominican Republic" "8 km NE of Hakuba, Japan" "177 km NNE of Saipan, Northern Mariana Islands" "17 km SE of Waynoka, Oklahoma" ...
##  $ type           : chr  "earthquake" "earthquake" "earthquake" "earthquake" ...
##  $ horizontalError: num  2.7 3.3 8.8 1.2 10.1 6.6 5.7 9.6 NA 14.4 ...
##  $ depthError     : num  7.1 4.1 6.8 1.7 9.5 8.7 5.7 1.9 0.3 1.9 ...
##  $ magError       : num  0 0.045 0.096 NA 0.153 0.13 0.201 0.059 NA 0.032 ...
##  $ magNst         : int  15 147 31 12 12 16 7 85 NA 301 ...
##  $ status         : chr  "reviewed" "reviewed" "reviewed" "reviewed" ...
##  $ locationSource : chr  "pr" "us" "us" "tul" ...
##  $ magSource      : chr  "pr" "us" "us" "slm" ...
\end{verbatim}

\hypertarget{the-parameters-and-their-explanations-for-this-data-are-given-below}{%
\paragraph{The parameters and their explanations for this data are given
below:}\label{the-parameters-and-their-explanations-for-this-data-are-given-below}}

\begin{longtable}[]{@{}
  >{\raggedright\arraybackslash}p{(\columnwidth - 2\tabcolsep) * \real{0.15}}
  >{\raggedright\arraybackslash}p{(\columnwidth - 2\tabcolsep) * \real{0.85}}@{}}
\toprule
\begin{minipage}[b]{\linewidth}\raggedright
Param Name
\end{minipage} & \begin{minipage}[b]{\linewidth}\raggedright
Description
\end{minipage} \\
\midrule
\endhead
time & Time when the event occurred. Times are reported in milliseconds
since the epoch \\
latitude & Decimal degrees latitude. Negative values for southern
latitudes \\
longitude & Decimal degrees longitude. Negative values for western
longitudes \\
depth & Depth of the event in kilometers \\
mag & The magnitude for the event \\
magType & The method or algorithm used to calculate the preferred
magnitude for the event \\
nst & The total number of seismic stations used to determine earthquake
location \\
gap & The largest azimuthal gap between azimuthally adjacent stations
(in degrees) \\
dmin & Horizontal distance from the epicenter to the nearest station (in
degrees) \\
rms & The root-mean-square (RMS) travel time residual, in sec, using all
weights \\
net & The ID of a data contributor \\
id & A unique identifier for the event \\
updated & Time when the event was most recently updated \\
place & Textual description of named geographic region near to the
event \\
type & A comma-separated list of product types associated to this
event \\
horizontalError & Uncertainty of reported location of the event in
kilometers \\
depthError & Uncertainty of reported depth of the event in kilometers \\
magError & Uncertainty of reported magnitude of the event \\
magNst & The total number of seismic stations used to calculate the
magnitude for this earthquake \\
status & Indicates whether the event has been reviewed by a human \\
locationSource & The network that originally authored the reported
location of this event \\
magSource & Network that originally authored the reported magnitude for
this event \\
\bottomrule
\end{longtable}

\hypertarget{checking-the-earthquake-types}{%
\paragraph{Checking the earthquake
types}\label{checking-the-earthquake-types}}

Since the data which is retrieved from USGS contains all natural
disasters, the type contains earthquake is filtered for our work. First,
we need to check type column contains our work interest which is the
earthquake.

\begin{Shaded}
\begin{Highlighting}[]
\FunctionTok{unique}\NormalTok{(data}\SpecialCharTok{$}\NormalTok{type)}
\end{Highlighting}
\end{Shaded}

\begin{verbatim}
##  [1] "earthquake"                 "mining explosion"          
##  [3] "other event"                "experimental explosion"    
##  [5] "explosion"                  "mine collapse"             
##  [7] "rock burst"                 "quarry blast"              
##  [9] "nuclear explosion"          "ice quake"                 
## [11] "landslide"                  "sonic boom"                
## [13] "collapse"                   "volcanic eruption"         
## [15] "induced or triggered event" "Ice Quake"
\end{verbatim}

\hypertarget{we-only-interest-with-earthquakes-so-we-have-to-remove-the-others-from-the-dataset}{%
\paragraph{we only interest with earthquakes so we have to remove the
others from the
dataset}\label{we-only-interest-with-earthquakes-so-we-have-to-remove-the-others-from-the-dataset}}

\hypertarget{in-this-way-we-get-rid-of-2k-rows-of-unnecessary-data.}{%
\paragraph{In this way, we get rid of 2k rows of unnecessary
data.}\label{in-this-way-we-get-rid-of-2k-rows-of-unnecessary-data.}}

\begin{Shaded}
\begin{Highlighting}[]
\NormalTok{data }\OtherTok{\textless{}{-}} \FunctionTok{filter}\NormalTok{(data, type}\SpecialCharTok{==}\StringTok{"earthquake"}\NormalTok{)}
\end{Highlighting}
\end{Shaded}

\hypertarget{lets-classify-earthquakes-according-to-their-magnitudes-in-order-to-use-them-in-graphs.}{%
\paragraph{Let's classify earthquakes according to their magnitudes in
order to use them in
graphs.}\label{lets-classify-earthquakes-according-to-their-magnitudes-in-order-to-use-them-in-graphs.}}

\begin{Shaded}
\begin{Highlighting}[]
\CommentTok{\#KOERI}
\NormalTok{turkey\_tidyquake }\OtherTok{\textless{}{-}}\NormalTok{ turkey\_tidyquake }\SpecialCharTok{\%\textgreater{}\%} 
                      \FunctionTok{mutate}\NormalTok{(}\AttributeTok{magClass =} \FunctionTok{cut}\NormalTok{(Mag, }\AttributeTok{breaks=}\FunctionTok{c}\NormalTok{(}\FloatTok{2.4}\NormalTok{,}\DecValTok{4}\NormalTok{,}\DecValTok{5}\NormalTok{,}\DecValTok{6}\NormalTok{,}\DecValTok{7}\NormalTok{,}\DecValTok{9}\NormalTok{),}
                                            \AttributeTok{labels=}\FunctionTok{c}\NormalTok{(}\StringTok{"2.5{-}4"}\NormalTok{, }\StringTok{"4{-}5"}\NormalTok{, }\StringTok{"5{-}6"}\NormalTok{, }\StringTok{"6{-}7"}\NormalTok{, }\StringTok{"7{-}9"}\NormalTok{)))}
\CommentTok{\#USGS}
\NormalTok{data}\OtherTok{\textless{}{-}} \FunctionTok{mutate}\NormalTok{(data, }\AttributeTok{magClass=}\FunctionTok{cut}\NormalTok{(data}\SpecialCharTok{$}\NormalTok{mag, }\AttributeTok{breaks=}\FunctionTok{c}\NormalTok{(}\FloatTok{2.4}\NormalTok{, }\DecValTok{4}\NormalTok{, }\DecValTok{5}\NormalTok{, }\DecValTok{6}\NormalTok{, }\DecValTok{7}\NormalTok{, }\DecValTok{9}\NormalTok{), }\AttributeTok{labels=}\FunctionTok{c}\NormalTok{(}\StringTok{"2.5{-}4"}\NormalTok{, }\StringTok{"4{-}5"}\NormalTok{, }\StringTok{"5{-}6"}\NormalTok{, }\StringTok{"6{-}7"}\NormalTok{, }\StringTok{"7{-}9"}\NormalTok{)))}
\end{Highlighting}
\end{Shaded}

\hypertarget{lets-format-the-time-column-and-separate-it-as-year-month-date}{%
\paragraph{Let's format the time column and separate it as
year-month-date}\label{lets-format-the-time-column-and-separate-it-as-year-month-date}}

\begin{Shaded}
\begin{Highlighting}[]
\CommentTok{\#KOERI}
\NormalTok{turkey\_tidyquake }\OtherTok{\textless{}{-}}\NormalTok{ turkey\_tidyquake }\SpecialCharTok{\%\textgreater{}\%} 
                      \FunctionTok{mutate}\NormalTok{(}\AttributeTok{Year =} \FunctionTok{year}\NormalTok{(Date),}
                             \AttributeTok{Month =} \FunctionTok{month}\NormalTok{(Date))}

\CommentTok{\#USGS}
\NormalTok{data}\SpecialCharTok{$}\NormalTok{time }\OtherTok{\textless{}{-}} \FunctionTok{strptime}\NormalTok{(data}\SpecialCharTok{$}\NormalTok{time, }\AttributeTok{format =} \StringTok{"\%Y{-}\%m{-}\%dT\%H:\%M:\%OSZ"}\NormalTok{)}
\NormalTok{data}\SpecialCharTok{$}\NormalTok{year }\OtherTok{\textless{}{-}} \FunctionTok{format}\NormalTok{(data}\SpecialCharTok{$}\NormalTok{time, }\AttributeTok{format=}\StringTok{"\%Y"}\NormalTok{)}
\NormalTok{data}\SpecialCharTok{$}\NormalTok{month }\OtherTok{\textless{}{-}} \FunctionTok{format}\NormalTok{(data}\SpecialCharTok{$}\NormalTok{time, }\AttributeTok{format=}\StringTok{"\%m"}\NormalTok{)}
\NormalTok{data}\SpecialCharTok{$}\NormalTok{date }\OtherTok{\textless{}{-}} \FunctionTok{format}\NormalTok{(data}\SpecialCharTok{$}\NormalTok{time, }\AttributeTok{format=}\StringTok{"\%Y\%m\%d"}\NormalTok{)}
\end{Highlighting}
\end{Shaded}

\hypertarget{quick-stats-summary-check-at-numeric-variables}{%
\paragraph{Quick stats summary check at numeric
variables}\label{quick-stats-summary-check-at-numeric-variables}}

\begin{Shaded}
\begin{Highlighting}[]
\FunctionTok{sapply}\NormalTok{(data[,}\FunctionTok{names}\NormalTok{(}\FunctionTok{which}\NormalTok{(}\FunctionTok{sapply}\NormalTok{(data, class) }\SpecialCharTok{==} \StringTok{"numeric"}\NormalTok{))],summary)}
\end{Highlighting}
\end{Shaded}

\begin{verbatim}
## $latitude
##     Min.  1st Qu.   Median     Mean  3rd Qu.     Max. 
## -82.8837   0.0914  19.4058  20.2736  42.3552  87.3860 
## 
## $longitude
##    Min. 1st Qu.  Median    Mean 3rd Qu.    Max. 
## -180.00 -155.26  -94.22  -49.60   74.00  180.00 
## 
## $depth
##    Min. 1st Qu.  Median    Mean 3rd Qu.    Max. 
##   -3.60    9.00   11.91   55.31   46.62  679.12 
## 
## $mag
##    Min. 1st Qu.  Median    Mean 3rd Qu.    Max. 
##   2.500   2.800   3.600   3.661   4.400   8.200 
## 
## $gap
##    Min. 1st Qu.  Median    Mean 3rd Qu.    Max.    NA's 
##     7.0    66.0   112.0   129.7   186.0   359.0   14300 
## 
## $dmin
##    Min. 1st Qu.  Median    Mean 3rd Qu.    Max.    NA's 
##   0.000   0.148   0.961   2.183   2.586 127.420   18523 
## 
## $rms
##    Min. 1st Qu.  Median    Mean 3rd Qu.    Max.    NA's 
##  0.0000  0.2800  0.5900  0.5932  0.8400 46.2400       5 
## 
## $horizontalError
##    Min. 1st Qu.  Median    Mean 3rd Qu.    Max.    NA's 
##   0.000   1.300   5.700   5.819   8.800  99.000   14206 
## 
## $depthError
##     Min.  1st Qu.   Median     Mean  3rd Qu.     Max.     NA's 
##    0.000    0.800    2.000    4.737    7.200 2329.800        7 
## 
## $magError
##    Min. 1st Qu.  Median    Mean 3rd Qu.    Max.    NA's 
##   0.000   0.078   0.129   0.249   0.200   5.670   22369
\end{verbatim}

\hypertarget{distribution-of-earthquakes-on-the-world-map-according-to-their-magnitude}{%
\paragraph{Distribution of earthquakes on the world map according to
their
magnitude}\label{distribution-of-earthquakes-on-the-world-map-according-to-their-magnitude}}

\begin{Shaded}
\begin{Highlighting}[]
\CommentTok{\#World map}
\NormalTok{p }\OtherTok{\textless{}{-}} \FunctionTok{ggplot}\NormalTok{() }\SpecialCharTok{+} \FunctionTok{geom\_map}\NormalTok{(}\AttributeTok{data =} \FunctionTok{map\_data}\NormalTok{(}\StringTok{"world"}\NormalTok{), }\AttributeTok{map =} \FunctionTok{map\_data}\NormalTok{(}\StringTok{"world"}\NormalTok{), }\FunctionTok{aes}\NormalTok{(}\AttributeTok{x =}\NormalTok{ long, }\AttributeTok{y=}\NormalTok{lat, }\AttributeTok{group=}\NormalTok{group, }\AttributeTok{map\_id=}\NormalTok{region), }\AttributeTok{fill=}\StringTok{"white"}\NormalTok{, }\AttributeTok{colour=}\StringTok{"\#7f7f7f"}\NormalTok{, }\AttributeTok{size=}\FloatTok{0.5}\NormalTok{)}
\end{Highlighting}
\end{Shaded}

\begin{verbatim}
## Warning: Ignoring unknown aesthetics: x, y
\end{verbatim}

\begin{Shaded}
\begin{Highlighting}[]
\CommentTok{\#Earthquakes points}
\NormalTok{p }\OtherTok{\textless{}{-}}\NormalTok{ p }\SpecialCharTok{+} \FunctionTok{geom\_point}\NormalTok{(}\AttributeTok{data =}\NormalTok{ data, }\FunctionTok{aes}\NormalTok{(}\AttributeTok{x=}\NormalTok{longitude, }\AttributeTok{y =}\NormalTok{ latitude, }\AttributeTok{colour =}\NormalTok{ mag)) }\SpecialCharTok{+} \FunctionTok{scale\_colour\_gradient}\NormalTok{(}\AttributeTok{low =} \StringTok{"\#00AA00"}\NormalTok{,}\AttributeTok{high =} \StringTok{"red"}\NormalTok{)}
\NormalTok{p}
\end{Highlighting}
\end{Shaded}

\includegraphics{project_final_NA_files/figure-latex/unnamed-chunk-15-1.pdf}

\hypertarget{distribution-of-earthquakes-on-turkey-according-to-their-magnitude}{%
\paragraph{Distribution of earthquakes on Turkey according to their
magnitude}\label{distribution-of-earthquakes-on-turkey-according-to-their-magnitude}}

Spatial map of Turkey is obtained from GADM(Database of Global
Administrative Areas)

\begin{Shaded}
\begin{Highlighting}[]
\CommentTok{\#url\_turkey \textless{}{-} "https://biogeo.ucdavis.edu/data/gadm3.6/Rsp/gadm36\_TUR\_1\_sp.rds"}
\CommentTok{\#download.file(url\_turkey, "data/sp\_turkey.rds")}

\NormalTok{tr\_sp }\OtherTok{\textless{}{-}} \FunctionTok{readRDS}\NormalTok{(}\StringTok{"data/sp\_turkey.rds"}\NormalTok{)}
\FunctionTok{plot}\NormalTok{(tr\_sp)}
\end{Highlighting}
\end{Shaded}

\begin{verbatim}
## Warning in wkt(obj): CRS object has no comment
\end{verbatim}

\includegraphics{project_final_NA_files/figure-latex/unnamed-chunk-16-1.pdf}

\hypertarget{spatial-data-is-converted-to-data-frame-below.-provinces-are-named-below-name_1-column.}{%
\paragraph{Spatial data is converted to data frame below. Provinces are
named below NAME\_1
column.}\label{spatial-data-is-converted-to-data-frame-below.-provinces-are-named-below-name_1-column.}}

\begin{Shaded}
\begin{Highlighting}[]
\NormalTok{tr\_df }\OtherTok{\textless{}{-}}\NormalTok{ tr\_sp }\SpecialCharTok{\%\textgreater{}\%} 
          \FunctionTok{as\_tibble}\NormalTok{()}

\FunctionTok{head}\NormalTok{(tr\_df)}
\end{Highlighting}
\end{Shaded}

\begin{verbatim}
## # A tibble: 6 x 10
##   GID_0 NAME_0 GID_1   NAME_1  VARNAME_1 NL_NAME_1 TYPE_1 ENGTYPE_1 CC_1  HASC_1
##   <chr> <chr>  <chr>   <chr>   <chr>     <chr>     <chr>  <chr>     <chr> <chr> 
## 1 TUR   Turkey TUR.1_1 Adana   Seyhan    <NA>      Il     Province  <NA>  TR.AA 
## 2 TUR   Turkey TUR.2_1 Adiyam~ Adıyaman  <NA>      Il     Province  <NA>  TR.AD 
## 3 TUR   Turkey TUR.3_1 Afyon   Afyonkar~ <NA>      Il     Province  <NA>  TR.AF 
## 4 TUR   Turkey TUR.4_1 Agri    Ağri|Kar~ <NA>      Il     Province  <NA>  TR.AG 
## 5 TUR   Turkey TUR.5_1 Aksaray <NA>      <NA>      Il     Province  <NA>  TR.AK 
## 6 TUR   Turkey TUR.6_1 Amasya  <NA>      <NA>      Il     Province  <NA>  TR.AM
\end{verbatim}

\hypertarget{we-obtain-geographical-data-for-turkey.}{%
\paragraph{We obtain geographical data for
Turkey.}\label{we-obtain-geographical-data-for-turkey.}}

\begin{Shaded}
\begin{Highlighting}[]
\NormalTok{tr\_geo }\OtherTok{\textless{}{-}}\NormalTok{ tr\_sp }\SpecialCharTok{\%\textgreater{}\%} 
            \FunctionTok{fortify}\NormalTok{()}
\end{Highlighting}
\end{Shaded}

\begin{verbatim}
## Regions defined for each Polygons
\end{verbatim}

\begin{Shaded}
\begin{Highlighting}[]
\FunctionTok{head}\NormalTok{(tr\_geo)}
\end{Highlighting}
\end{Shaded}

\begin{verbatim}
##       long      lat order  hole piece id group
## 1 35.41454 36.58850     1 FALSE     1  1   1.1
## 2 35.41459 36.58820     2 FALSE     1  1   1.1
## 3 35.41434 36.58820     3 FALSE     1  1   1.1
## 4 35.41347 36.58820     4 FALSE     1  1   1.1
## 5 35.41347 36.58792     5 FALSE     1  1   1.1
## 6 35.41236 36.58792     6 FALSE     1  1   1.1
\end{verbatim}

\hypertarget{dataframe-that-contains-provinces-is-created-and-joint-to-geographical-data-for-turkey.}{%
\paragraph{Dataframe that contains provinces is created and joint to
geographical data for
Turkey.}\label{dataframe-that-contains-provinces-is-created-and-joint-to-geographical-data-for-turkey.}}

\begin{Shaded}
\begin{Highlighting}[]
\FunctionTok{library}\NormalTok{(stringi) }\CommentTok{\#This is for Turkish characters such as ü, ş, etc.}
\NormalTok{prov }\OtherTok{=} \FunctionTok{tibble}\NormalTok{(}\AttributeTok{id =} \FunctionTok{rownames}\NormalTok{(tr\_df), }\AttributeTok{province =} \FunctionTok{stri\_trans\_general}\NormalTok{(tr\_df}\SpecialCharTok{$}\NormalTok{NAME\_1, }\StringTok{"Latin{-}ASCII"}\NormalTok{))}
\NormalTok{tr\_geo\_final }\OtherTok{\textless{}{-}} \FunctionTok{left\_join}\NormalTok{(tr\_geo, prov, }\AttributeTok{by =} \StringTok{"id"}\NormalTok{)}
\NormalTok{tr\_geo\_final}\SpecialCharTok{$}\NormalTok{id }\OtherTok{\textless{}{-}} \FunctionTok{as.numeric}\NormalTok{(tr\_geo\_final}\SpecialCharTok{$}\NormalTok{id)}
\end{Highlighting}
\end{Shaded}

\hypertarget{using-our-geographical-data-we-can-draw-the-map-of-turkey-with-ggplot.}{%
\paragraph{Using our geographical data, we can draw the map of Turkey
with
ggplot.}\label{using-our-geographical-data-we-can-draw-the-map-of-turkey-with-ggplot.}}

\begin{Shaded}
\begin{Highlighting}[]
\NormalTok{tr\_map }\OtherTok{\textless{}{-}} \FunctionTok{ggplot}\NormalTok{()}\SpecialCharTok{+}
            \FunctionTok{theme\_minimal}\NormalTok{()}\SpecialCharTok{+}
            \CommentTok{\#Drawing the borders of Turkey and its provinces}
            \FunctionTok{geom\_polygon}\NormalTok{(}\AttributeTok{data =}\NormalTok{ tr\_geo\_final, }\FunctionTok{aes}\NormalTok{(}\AttributeTok{x=}\NormalTok{long, }\AttributeTok{y =}\NormalTok{ lat, }\AttributeTok{group =}\NormalTok{ group, }\AttributeTok{map\_id =}\NormalTok{ id),}
                         \AttributeTok{color =} \StringTok{"black"}\NormalTok{, }\AttributeTok{fill =} \StringTok{"white"}\NormalTok{) }\SpecialCharTok{+}
            \CommentTok{\#Fixed the scale of coordinate system}
            \FunctionTok{coord\_fixed}\NormalTok{()}
\end{Highlighting}
\end{Shaded}

\begin{verbatim}
## Warning: Ignoring unknown aesthetics: map_id
\end{verbatim}

\begin{Shaded}
\begin{Highlighting}[]
\NormalTok{tr\_map}
\end{Highlighting}
\end{Shaded}

\includegraphics{project_final_NA_files/figure-latex/unnamed-chunk-20-1.pdf}

\hypertarget{at-the-end-the-locations-of-the-earthquakes-are-pointed-on-the-map.}{%
\paragraph{At the end, the locations of the earthquakes are pointed on
the
map.}\label{at-the-end-the-locations-of-the-earthquakes-are-pointed-on-the-map.}}

\begin{Shaded}
\begin{Highlighting}[]
\NormalTok{tr\_map }\SpecialCharTok{+}
  \FunctionTok{labs}\NormalTok{(}\AttributeTok{title =} \StringTok{"Earthquake Map of Turkey in Years 2016{-}2020"}\NormalTok{)}\SpecialCharTok{+}
  \FunctionTok{theme}\NormalTok{(}\AttributeTok{plot.title =} \FunctionTok{element\_text}\NormalTok{(}\AttributeTok{hjust =} \FloatTok{0.5}\NormalTok{)) }\SpecialCharTok{+}
  \FunctionTok{geom\_point}\NormalTok{(}\AttributeTok{data =}\NormalTok{ turkey\_tidyquake, }\FunctionTok{aes}\NormalTok{(}\AttributeTok{x =}\NormalTok{ Longitude, }\AttributeTok{y =}\NormalTok{ Latitude, }\AttributeTok{color =}\NormalTok{ Mag), }\AttributeTok{alpha=}\FloatTok{0.7}\NormalTok{) }\SpecialCharTok{+}
  \FunctionTok{scale\_color\_gradient}\NormalTok{(}\AttributeTok{low =} \StringTok{"\#00AA00"}\NormalTok{, }\AttributeTok{high =} \StringTok{"red"}\NormalTok{)}
\end{Highlighting}
\end{Shaded}

\includegraphics{project_final_NA_files/figure-latex/unnamed-chunk-21-1.pdf}

As seen above, earthquakes are occurred often on northwest to southwest
of Turkey. Also, we can see the earthquakes are common on The North
Anatolian Fault.

\hypertarget{distribution-of-the-number-of-earthquakes-by-years.}{%
\paragraph{Distribution of The Number of Earthquakes by
Years.}\label{distribution-of-the-number-of-earthquakes-by-years.}}

As seen below, more than 20 thousand earthquakes occurred each year from
2016 to 2020. In 2018, there were nearly twice as many earthquakes
occurred, compared to 2017. This is the highest count of earthquakes in
these 5 years.

\begin{Shaded}
\begin{Highlighting}[]
\NormalTok{year}\OtherTok{\textless{}{-}}\NormalTok{data }\SpecialCharTok{\%\textgreater{}\%} \FunctionTok{group\_by}\NormalTok{(year) }\SpecialCharTok{\%\textgreater{}\%} \FunctionTok{tally}\NormalTok{()}
\NormalTok{p }\OtherTok{\textless{}{-}} \FunctionTok{ggplot}\NormalTok{(year) }\SpecialCharTok{+} \FunctionTok{geom\_bar}\NormalTok{(}\FunctionTok{aes}\NormalTok{(}\AttributeTok{x=}\NormalTok{year, }\AttributeTok{y=}\NormalTok{n, }\AttributeTok{fill =} \FunctionTok{as.factor}\NormalTok{(n)), }\AttributeTok{stat=}\StringTok{"identity"}\NormalTok{)}\SpecialCharTok{+}
      \FunctionTok{scale\_fill\_brewer}\NormalTok{(}\AttributeTok{palette =} \StringTok{"YlOrRd"}\NormalTok{)}\SpecialCharTok{+}
      \FunctionTok{theme\_minimal}\NormalTok{()}\SpecialCharTok{+}
      \FunctionTok{theme}\NormalTok{(}\AttributeTok{legend.position =} \StringTok{"none"}\NormalTok{)}
\NormalTok{p }\OtherTok{\textless{}{-}}\NormalTok{ p }\SpecialCharTok{+} \FunctionTok{ggtitle}\NormalTok{(}\StringTok{"Distribution of Earthquake Counts in The World by Years"}\NormalTok{) }\SpecialCharTok{+}
      \FunctionTok{xlab}\NormalTok{(}\StringTok{"Years"}\NormalTok{) }\SpecialCharTok{+} \FunctionTok{ylab}\NormalTok{(}\StringTok{"Counts"}\NormalTok{)}

\NormalTok{p}
\end{Highlighting}
\end{Shaded}

\includegraphics{project_final_NA_files/figure-latex/unnamed-chunk-22-1.pdf}

In Turkey, there are more than one thousand earthquakes happened each
year from 2016 to 2020. In 2017, more than 5000 of earthquakes occurred
in Turkey, which is a peak in these 5 years.

\begin{Shaded}
\begin{Highlighting}[]
\NormalTok{year\_turkey }\OtherTok{\textless{}{-}}\NormalTok{ turkey\_tidyquake }\SpecialCharTok{\%\textgreater{}\%}
                \FunctionTok{group\_by}\NormalTok{(Year) }\SpecialCharTok{\%\textgreater{}\%} 
                \FunctionTok{tally}\NormalTok{()}
\NormalTok{tr\_yearthquake }\OtherTok{\textless{}{-}} \FunctionTok{ggplot}\NormalTok{(year\_turkey) }\SpecialCharTok{+} \FunctionTok{geom\_bar}\NormalTok{(}\FunctionTok{aes}\NormalTok{(}\AttributeTok{x =}\NormalTok{ Year, }\AttributeTok{y =}\NormalTok{ n, }\AttributeTok{fill =} \FunctionTok{as.factor}\NormalTok{(n)),}
                                                 \AttributeTok{stat =} \StringTok{"identity"}\NormalTok{)}\SpecialCharTok{+}
                  \FunctionTok{scale\_fill\_brewer}\NormalTok{(}\AttributeTok{palette =} \StringTok{"Blues"}\NormalTok{)}\SpecialCharTok{+}
                  \FunctionTok{theme\_minimal}\NormalTok{()}\SpecialCharTok{+}
                  \FunctionTok{theme}\NormalTok{(}\AttributeTok{legend.position =} \StringTok{"none"}\NormalTok{)}\SpecialCharTok{+}
                  \FunctionTok{ylab}\NormalTok{(}\StringTok{"Counts"}\NormalTok{)}
\NormalTok{tr\_yearthquake }\OtherTok{\textless{}{-}}\NormalTok{ tr\_yearthquake }\SpecialCharTok{+} \FunctionTok{ggtitle}\NormalTok{(}\StringTok{"Distribution of Earthquake Counts in Turkey by Years"}\NormalTok{)}
\NormalTok{tr\_yearthquake}
\end{Highlighting}
\end{Shaded}

\includegraphics{project_final_NA_files/figure-latex/unnamed-chunk-23-1.pdf}

\hypertarget{distribution-of-the-number-of-earthquakes-by-months.}{%
\paragraph{Distribution of The Number of Earthquakes by
Months.}\label{distribution-of-the-number-of-earthquakes-by-months.}}

There are more than 10 thousand earthquakes observed in the world each
year from 2016 to 2020. The count of earthquakes increased on Summer.
Therefore there could be a relationship between temperature and the
earthquakes.

\begin{Shaded}
\begin{Highlighting}[]
\FunctionTok{library}\NormalTok{(RColorBrewer)}
\NormalTok{month}\OtherTok{\textless{}{-}}\NormalTok{data }\SpecialCharTok{\%\textgreater{}\%} \FunctionTok{group\_by}\NormalTok{(month) }\SpecialCharTok{\%\textgreater{}\%} \FunctionTok{tally}\NormalTok{()}
\NormalTok{world\_monthquake }\OtherTok{\textless{}{-}} \FunctionTok{ggplot}\NormalTok{(month) }\SpecialCharTok{+} \FunctionTok{geom\_bar}\NormalTok{(}\FunctionTok{aes}\NormalTok{(}\AttributeTok{x=}\NormalTok{month, }\AttributeTok{y=}\NormalTok{n, }\AttributeTok{fill=} \FunctionTok{as.factor}\NormalTok{(n)), }\AttributeTok{stat=}\StringTok{"identity"}\NormalTok{) }\SpecialCharTok{+}
     \FunctionTok{scale\_fill\_manual}\NormalTok{(}\AttributeTok{values =} \FunctionTok{colorRampPalette}\NormalTok{(}\FunctionTok{brewer.pal}\NormalTok{(}\DecValTok{9}\NormalTok{,}\StringTok{"PuRd"}\NormalTok{))(}\DecValTok{12}\NormalTok{))}\SpecialCharTok{+}
     \FunctionTok{theme\_minimal}\NormalTok{()}\SpecialCharTok{+}
     \FunctionTok{theme}\NormalTok{(}\AttributeTok{legend.position =} \StringTok{"none"}\NormalTok{)}
\NormalTok{world\_monthquake }\OtherTok{\textless{}{-}}\NormalTok{ world\_monthquake }\SpecialCharTok{+} \FunctionTok{ggtitle}\NormalTok{(}\StringTok{"Distribution of Earthquake Counts in The World by Months"}\NormalTok{) }\SpecialCharTok{+} \FunctionTok{xlab}\NormalTok{(}\StringTok{"Months"}\NormalTok{) }\SpecialCharTok{+} \FunctionTok{ylab}\NormalTok{(}\StringTok{"Counts"}\NormalTok{)}
\NormalTok{world\_monthquake}
\end{Highlighting}
\end{Shaded}

\includegraphics{project_final_NA_files/figure-latex/unnamed-chunk-24-1.pdf}
In Turkey, we also see the number of earthquakes are increased on Summer
season, while the most of earthquakes occurred in January and February.

\begin{Shaded}
\begin{Highlighting}[]
\NormalTok{tr\_monthquake }\OtherTok{\textless{}{-}}\NormalTok{ turkey\_tidyquake }\SpecialCharTok{\%\textgreater{}\%} 
                  \FunctionTok{group\_by}\NormalTok{(Month) }\SpecialCharTok{\%\textgreater{}\%} 
                  \FunctionTok{tally}\NormalTok{()}
\NormalTok{tr\_month\_plot }\OtherTok{\textless{}{-}} \FunctionTok{ggplot}\NormalTok{(tr\_monthquake) }\SpecialCharTok{+} \FunctionTok{geom\_bar}\NormalTok{(}\FunctionTok{aes}\NormalTok{(}\AttributeTok{x=}\NormalTok{Month, }\AttributeTok{y=}\NormalTok{n, }\AttributeTok{fill=} \FunctionTok{as.factor}\NormalTok{(n)),}
                                                  \AttributeTok{stat =} \StringTok{"identity"}\NormalTok{) }\SpecialCharTok{+}
                  \FunctionTok{scale\_fill\_manual}\NormalTok{(}\AttributeTok{values =} \FunctionTok{colorRampPalette}\NormalTok{(}\FunctionTok{brewer.pal}\NormalTok{(}\DecValTok{9}\NormalTok{,}\StringTok{"Oranges"}\NormalTok{))(}\DecValTok{12}\NormalTok{))}\SpecialCharTok{+}
                  \FunctionTok{scale\_x\_continuous}\NormalTok{(}\AttributeTok{breaks =} \FunctionTok{c}\NormalTok{(}\DecValTok{1}\SpecialCharTok{:}\DecValTok{12}\NormalTok{))}\SpecialCharTok{+}
                  \FunctionTok{theme\_minimal}\NormalTok{()}\SpecialCharTok{+}
                  \FunctionTok{theme}\NormalTok{(}\AttributeTok{legend.position =} \StringTok{"none"}\NormalTok{)}
\NormalTok{tr\_month\_plot }\OtherTok{\textless{}{-}}\NormalTok{ tr\_month\_plot }\SpecialCharTok{+} 
                  \FunctionTok{ggtitle}\NormalTok{(}\StringTok{"Distribution of Earthquake Counts in Turkey by Months"}\NormalTok{)}
\NormalTok{tr\_month\_plot}
\end{Highlighting}
\end{Shaded}

\includegraphics{project_final_NA_files/figure-latex/unnamed-chunk-25-1.pdf}

\hypertarget{distribution-of-the-earthquakes-by-years-and-magclasses.}{%
\paragraph{Distribution of The Earthquakes by years and
magClasses.}\label{distribution-of-the-earthquakes-by-years-and-magclasses.}}

\begin{Shaded}
\begin{Highlighting}[]
\NormalTok{year}\OtherTok{\textless{}{-}}\NormalTok{data }\SpecialCharTok{\%\textgreater{}\%} \FunctionTok{group\_by}\NormalTok{(year, magClass) }\SpecialCharTok{\%\textgreater{}\%} \FunctionTok{tally}\NormalTok{()}
\NormalTok{year}\SpecialCharTok{\%\textgreater{}\%}\FunctionTok{ggplot}\NormalTok{(}\FunctionTok{aes}\NormalTok{(year, n))}\SpecialCharTok{+}
\FunctionTok{geom\_point}\NormalTok{(}\AttributeTok{size=}\DecValTok{1}\NormalTok{, }\AttributeTok{col=}\StringTok{"red"}\NormalTok{)}\SpecialCharTok{+}
  \FunctionTok{facet\_wrap}\NormalTok{(}\SpecialCharTok{\textasciitilde{}}\NormalTok{magClass,  }\AttributeTok{ncol=}\DecValTok{2}\NormalTok{, }\AttributeTok{scales=}\StringTok{"free"}\NormalTok{)}\SpecialCharTok{+}
   \FunctionTok{ggtitle}\NormalTok{(}\StringTok{"Number of Earthquakes by Magnitude Class and Year"}\NormalTok{) }\SpecialCharTok{+}
           \FunctionTok{xlab}\NormalTok{(}\StringTok{"Year"}\NormalTok{) }\SpecialCharTok{+} \FunctionTok{ylab}\NormalTok{(}\StringTok{"Number of Cases"}\NormalTok{)}\SpecialCharTok{+}
  \FunctionTok{theme}\NormalTok{(}\AttributeTok{plot.title =} \FunctionTok{element\_text}\NormalTok{(}\AttributeTok{face=}\StringTok{"bold"}\NormalTok{, }\AttributeTok{size=}\DecValTok{14}\NormalTok{, }\AttributeTok{hjust=}\FloatTok{0.5}\NormalTok{)) }\SpecialCharTok{+}
\FunctionTok{theme}\NormalTok{(}\AttributeTok{axis.title =} \FunctionTok{element\_text}\NormalTok{(}\AttributeTok{face=}\StringTok{"bold"}\NormalTok{, }\AttributeTok{size=}\DecValTok{12}\NormalTok{))}
\end{Highlighting}
\end{Shaded}

\includegraphics{project_final_NA_files/figure-latex/unnamed-chunk-26-1.pdf}

\hypertarget{lets-examine-the-distribution-of-earthquakes-on-earth-by-months-and-magclasses.}{%
\paragraph{Let's examine the distribution of earthquakes on Earth by
months and
magClasses.}\label{lets-examine-the-distribution-of-earthquakes-on-earth-by-months-and-magclasses.}}

\begin{Shaded}
\begin{Highlighting}[]
\NormalTok{month}\OtherTok{\textless{}{-}}\NormalTok{data }\SpecialCharTok{\%\textgreater{}\%} \FunctionTok{group\_by}\NormalTok{(month, magClass) }\SpecialCharTok{\%\textgreater{}\%} \FunctionTok{tally}\NormalTok{()}
\NormalTok{month}\SpecialCharTok{\%\textgreater{}\%}\FunctionTok{ggplot}\NormalTok{(}\FunctionTok{aes}\NormalTok{(month, n))}\SpecialCharTok{+}
\FunctionTok{geom\_point}\NormalTok{(}\AttributeTok{size=}\DecValTok{1}\NormalTok{, }\AttributeTok{col=}\StringTok{"red"}\NormalTok{)}\SpecialCharTok{+}
  \FunctionTok{facet\_wrap}\NormalTok{(}\SpecialCharTok{\textasciitilde{}}\NormalTok{magClass,  }\AttributeTok{ncol=}\DecValTok{2}\NormalTok{, }\AttributeTok{scales=}\StringTok{"free"}\NormalTok{)}\SpecialCharTok{+}
   \FunctionTok{ggtitle}\NormalTok{(}\StringTok{"Number of Earthquakes by Magnitude Class and Month"}\NormalTok{) }\SpecialCharTok{+}
           \FunctionTok{xlab}\NormalTok{(}\StringTok{"Months"}\NormalTok{) }\SpecialCharTok{+} \FunctionTok{ylab}\NormalTok{(}\StringTok{"Number of Cases"}\NormalTok{)}\SpecialCharTok{+}
  \FunctionTok{theme}\NormalTok{(}\AttributeTok{plot.title =} \FunctionTok{element\_text}\NormalTok{(}\AttributeTok{face=}\StringTok{"bold"}\NormalTok{, }\AttributeTok{size=}\DecValTok{14}\NormalTok{, }\AttributeTok{hjust=}\FloatTok{0.5}\NormalTok{)) }\SpecialCharTok{+}
\FunctionTok{theme}\NormalTok{(}\AttributeTok{axis.title =} \FunctionTok{element\_text}\NormalTok{(}\AttributeTok{face=}\StringTok{"bold"}\NormalTok{, }\AttributeTok{size=}\DecValTok{12}\NormalTok{))}
\end{Highlighting}
\end{Shaded}

\includegraphics{project_final_NA_files/figure-latex/unnamed-chunk-27-1.pdf}

\hypertarget{lets-check-the-density-function-of-magnitude.}{%
\paragraph{Let's check the density function of
Magnitude.}\label{lets-check-the-density-function-of-magnitude.}}

\begin{Shaded}
\begin{Highlighting}[]
\FunctionTok{ggdensity}\NormalTok{(data}\SpecialCharTok{$}\NormalTok{mag, }
          \AttributeTok{main =} \StringTok{"Density plot of magnitude"}\NormalTok{,}
          \AttributeTok{xlab =} \StringTok{"Magnitute"}\NormalTok{)}
\end{Highlighting}
\end{Shaded}

\includegraphics{project_final_NA_files/figure-latex/unnamed-chunk-28-1.pdf}

\hypertarget{results-and-discussion}{%
\subsection{Results and Discussion}\label{results-and-discussion}}

As a result of our research, we have reached: We learned the averages of
earthquake magnitudes in the world. On the world map, we have seen that
earthquakes are more intense on coastlines. Compared to other years, we
saw that the number of earthquakes in 2018 was almost doubled. We
observed an increase in the number of earthquakes in the summer months.
This increase seems especially 2.5-4 magnitude range.

Thanks to our project, we obtained separate insights from earthquake
data of the World and Turkey. We checked whether the earthquake
relations are a link between the world and Turkey. We compared the world
averages with Turkey, which is known as the earthquake zone. In this
process, we determined that we could make comparisons on the basis of
size, hourly, seasonal, seasonal (sea/terrestrial) and we focused on
these factors in our research accordingly.

\hypertarget{conclusion}{%
\subsection{Conclusion}\label{conclusion}}

As a result, we used data transfer, cleaning, reconstruction according
to data types and basic visualization processes in the analysis of
earthquake data in the world and Turkey between the years 2016-2020,
which we obtained from USGS and KOERI organizations, which we identified
as reliable data sources. In this way, we provided the opportunity to
visually see whether there is a similarity between the earthquakes that
took place in the world and in Turkey. In this way, we tried to figure
out whether our country, which we refer to as an earthquake zone, is
really an above-average earthquake zone when compared to other countries
in the world.

You can also access our project's GitHub page here:
\href{https://github.com/MAT381E-Fall21/project_final_report-na}{Statistics
of earthquake hazards in Turkey and comparison with the world}

\hypertarget{references}{%
\subsection{References}\label{references}}

\begin{itemize}
\tightlist
\item
  \url{https://earthquake.usgs.gov/earthquakes/search/}
\item
  \url{https://earthquake.usgs.gov/data/comcat/index.php}
\item
  \url{http://www.koeri.boun.edu.tr/sismo/2/earthquake-catalog/}
\item
  \url{https://towardsdatascience.com/reverse-geocoding-in-r-f7fe4b908355}
\item
  \url{https://tevfikbulut.com/2020/02/02/exploratory-data-analysis-of-turkey-earthquakes-ii/}
\item
  \url{https://www.r-bloggers.com/2019/04/earthquake-analysis-1-4-quantitative-variables-exploratory-analysis/}
\end{itemize}

\end{document}
